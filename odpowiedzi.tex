\documentclass[a4paper]{article}
\usepackage[T1]{fontenc}
\usepackage[utf8]{inputenc}
\usepackage{lmodern}
\usepackage{amssymb}

%~ \usepackage{enumerate}
\usepackage{enumitem}
%~ \usepackage[margin=1in]{geometry}
\usepackage{geometry}
\usepackage{multicol}

\usepackage{color}
\definecolor{gray}{rgb}{0.8, 0.8, 0.8}
\definecolor{deepgreen}{rgb}{0.156, 0.43, 0}

\usepackage{listings}

\lstset{
language=Python,
basicstyle=\ttfamily,
keywordstyle=\bfseries,
tabsize=4,
numbers=left,
numberblanklines=false,
showstringspaces=false,
keepspaces,
keywordstyle=\color{blue},
stringstyle=\color{deepgreen}
}

\lstset{
literate={ą}{{\c a}}1
{ć}{{\' c }}1
{ę}{{\c e }}1
{ł}{{\l}}1
{ń}{{\'n}}1
{ó}{{\'o}}1
{ś}{{\'s}}1
{ź}{{\'z}}1
{ż}{{\.z}}1
}

\usepackage{polski}

\usepackage{graphicx}
\usepackage{longtable}
\usepackage{hyperref}

% makro nadające tło 'gray' tekstowi podanemu jako parametr - stara wersja
% \newcommand{\important}[1]{\colorbox{gray}{\begin{minipage}[t]{\linewidth}#1\end{minipage}}}

% \newcommand{\important}[1]{\colorbox{gray}{\parbox[t]{0.9\linewidth}{#1}}}

\newcommand{\important}[1]{
    \begin{center}\colorbox{gray}{
        \begin{minipage}[t]{0.9\textwidth}{#1}
        \end{minipage}
    }
    \end{center}
}

\newcommand{\kwords}[1]{\begin{center}\colorbox{gray}{\parbox{0.9\textwidth}{\textbf{Słowa kluczowe:} \emph{#1}}}\end{center}\vspace{1eM}}

\newcommand{\key}[1]{\textbf{#1}}

%~ \usepackage[polish]{babel}
%~ \selectlanguage{polish}

\setlength{\parindent}{0pt}
\setlength{\parskip}{1ex plus 0.5ex minus 0.2ex}

\title{Odpowiedzi do wybranych zadań}
%~ \author{Wojciech Świderski}

\begin{document}
\maketitle
\tableofcontents

% --- czy counter MUSI być zdefiniowany tu, czy może być w preambule?
\newcounter{zadanie}
\setcounter{zadanie}{1}
%------------------------------

\section{Wejście / wyjście}
\textbf{\arabic{zadanie}.}\addtocounter{zadanie}{1}

\begin{tabular}{l|l|l}
\textsf{rodzaj danych} & \textsf{zachęta} & \textsf{input} \\\hline
liczba całkowita & \dots & \verb|d = int(input('długość w cm: '))|\\\hline
liczba całkowita & \dots & \verb|d = int(input('waga w kg: '))|\\\hline
wartość logiczna & \dots & \verb|ans = bool(input('tak / nie? '))|\\\hline
tekst & \dots & \verb|imie = input('Podaj imię: ')| \\\hline
liczba rzeczywista & \dots & \verb|szer = float(input(('Podaj szerokość: '))| \\\hline
liczba całkowita & \dots & \verb|l_g = int(input('liczba gości: '))| \\\hline
tekst & \dots & \verb|nazwisko = input('nazwisko: ')| \\\hline
liczba całkowita & \dots & \verb|wiek = int(input('wiek w latach: '))| \\\hline
liczba całkowita & \dots & \verb|liczba_pok = int(input('liczba pokojów: '))| \\\hline
liczba rzeczywista & \dots & \verb|dlugosc = float(input('długość w metrach: '))| \\\hline
wartość logiczna & \dots & \verb|czy_wiecej = bool(input('więcej? '))|
\end{tabular}

\textbf{\arabic{zadanie}.}\addtocounter{zadanie}{1}
\begin{multicols}{2}
\begin{enumerate}[label=\arabic*.]
    \item \verb|print(e, f, sep='/')|
    \item \verb|print(e, f, sep=' / ')|
    \item \verb|print(e, f, sep=', ')|
    \item \verb|print(e, f, sep=',\n')|
    \item \verb|print(str, str_2, sep=' i ')|
    \item \verb|print(str, str_2, sep=', ')|
    \item \verb|print(str, str_2, sep='\n')|
    \item \verb|print(str, str_2, sep='\t')|
\end{enumerate}
\end{multicols}

\section{Zmienne}
\textbf{\arabic{zadanie}.}\addtocounter{zadanie}{1} Nazwa zmiennej \emph{nie może} zwierać spacji, znaków interpunkcyjnych i znaków operacji matemtycznych. Nie może się zaczynać od cyfry, choć może cyfry zawierać. \verb|_| (,,podłoga'') jest dozwolona jako nazwa zmiennej.

W nazwach zmiennych \emph{mogą} występować znaki specyficzne dla niektórych języków (np. polskiego). \emph{Nie jest} to jednak zalecane.

\begin{multicols}{3}
\begin{enumerate}[label=\arabic*.]
    \item tak
    \item tak
    \item tak
    \item tak
    \item tak
    \item tak
    \item nie
    \item tak
    \item tak
    \item tak
    \item nie
    \item nie
    \item nie
    \item nie
    \item nie
    \item nie
    \item nie
    \item nie
    \item tak
    \item tak
    \item tak
    \item tak
    \item tak
    \item tak
    \item tak
    \item tak
    \item tak
    \item nie
    \item tak
    \item tak
    \item nie
    \item tak
    \item tak
    \item tak
    \item nie
    \item nie
    \item tak
    \item tak
    \item nie
    \item nie
\end{enumerate}
\end{multicols}

\textbf{\arabic{zadanie}.}\addtocounter{zadanie}{1}

\begin{enumerate}[label=\arabic*.]
    \item \verb|l = 4|
    \item \verb|i = -1|
    \item \verb|j = 2 * 5|
    \item \verb|k = 2 / 5|
    \item \verb|r = 2.4|
    \item \verb|b = 0b1000|
    \item \verb|o = 0o11|
    \item \verb|h = 0h10|
    \item \verb|test = False|
    \item \verb|jezyk = 'Python'|
    \item \verb|ver = '3.10'|
    \item \verb|i2 = 3 ** 2|
    \item \verb|imie = 'Adam'|
    \item \verb|kody = {'a': 0, 'b': 1, 'c': 3, 'd': 4}|
    \item \verb|odwiedzone = {1, 2, 3, 4, 5}|
    \item \verb|pierwsze = (2, 3, 5, 7)|
    \item \verb|tic_tac = [[0, 0, 0], [0, 0, 0], [0, 0, 0]]|
\end{enumerate}

\textbf{\arabic{zadanie}.}\addtocounter{zadanie}{1}

\begin{multicols}{2}
\begin{enumerate}[label=\arabic*.]
    \item \verb|l = 2 * l|
    \item \verb|i = -i|
    \item \verb|j = j ** 2 + 5|
    \item \verb|k = k / 5|
    \item \verb|r = 2.4 * r|
    \item \verb|test = not test|
    \item \verb|jezyk = 'Nim'|
    \item \verb|ver = '1.60'|
    \item \verb|o = 0o123|
    \item \verb|h = 0hfe|
    \item \verb|b = 0b111|
    \item \verb|imie = 'Ewa'|
\end{enumerate}
\end{multicols}

\textbf{\arabic{zadanie}.}\addtocounter{zadanie}{1}

\verb|t = 4|

\begin{enumerate}[label=\arabic*.]
    \item \verb|t += 1|
    \item \verb|t *= 4|
    \item \verb|t //= 2|
    \item \verb|t -= 3|
    \item \verb|t //= 3|
    \item \verb|t **= 3|
    \item \verb|t %= 5|
\end{enumerate}

Ostatecznie \verb|t = 3|.

\section{Wyrażenia}

\textbf{\arabic{zadanie}.}\addtocounter{zadanie}{1}

\begin{multicols}{3}
\begin{enumerate}[label=\arabic*.]
    \item \verb|3, int|
    \item \verb|3.0, float|
    \item \verb|50, int|
    \item \verb|50.0, float|
    \item \verb|1.0, float|
    \item \verb|2.0, float|
    \item \verb|0.5, float|
    \item \verb|0, int|
    \item \verb|0, int|
    \item \verb|2, int|
    \item \verb|2.0, float|
    \item \verb|0, int|
    \item \verb|0.0, float|
    \item \verb|1.0, float|
    \item \verb|0.7, float|
    \item \verb|0.0, float|
    \item \verb|2, int|
    \item \verb|2, int|

    \item \verb|5, int|
    \item \verb|3, int|
    \item \verb|0, int|
    \item \verb|4.0, float|
    \item \verb|0.75, float|
    \item \verb|0.75, float|
    \item \verb|0, float|
    \item \verb|0, float|

    \item \verb|1.0, float|

    \item \verb|8, int|
    \item \verb|2.8284271247461903, float|
    \item \verb|2, int|
    \item \verb|7, int|
    \item \verb|3, int|
    \item \verb|3, int|
    \item \verb|(2, 3), tuple|
    \item \verb|(5.0, 1.0), tuple|
    \item \verb|(5.0, 1.0), tuple|

\end{enumerate}
\end{multicols}

\textbf{\arabic{zadanie}.}\addtocounter{zadanie}{1} Typ wyniku podany obok wartości.

\begin{multicols}{2}
\begin{enumerate}[label=\arabic*.]
    \item \verb|1, int|
    \item \verb|3, int|
    \item \verb|0.0, float|
    \item \verb|0.0, float|
    \item \verb|0, int|
    \item \verb|1, int|
    \item \verb|2.0, float|
    \item \verb|1, int|
    \item \verb|1.0, float|
    \item \verb|True, bool|
    \item \verb|False, bool|
    \item \verb|True, bool|
    \item \verb|False, bool|
    \item \verb|0, int|
    \item \verb|True, bool|
    \item \verb|False, bool|

\end{enumerate}
\end{multicols}

\textbf{\arabic{zadanie}.}\addtocounter{zadanie}{1}

\begin{multicols}{2}
\begin{enumerate}[label=\arabic*.]
    \item \verb|True|
    \item \verb|True|
    \item \verb|True|
    \item \verb|True|
    \item \verb|True|
    \item \verb|False|
    \item \verb|True|
    \item \verb|False|
    \item \verb|True|
    \item \verb|True|
    \item \verb|True|
    \item \verb|False|
    \item \verb|True|
    \item \verb|False|
    \item \verb|True|
    \item \verb|True|
    \item \verb|True|
    \item \verb|False|
    \item \verb|True|
    \item \verb|True|
    \item \verb|True|
    \item \verb|False|
\end{enumerate}
\end{multicols}

\textbf{\arabic{zadanie}.}\addtocounter{zadanie}{1}

\begin{multicols}{2}
\begin{enumerate}[label=\arabic*.]
    \item \verb|True|
    \item \verb|'a'|
    \item \verb|2|
    \item \verb|''|
    \item \verb|'kot'|
    \item \verb|3|
    \item \verb|2|
    \item \verb|4|
    \item \verb|[]|
    \item \verb|2|

\end{enumerate}

\end{multicols}

\section{Rozpakowanie}

\textbf{\arabic{zadanie}.}\addtocounter{zadanie}{1}

\begin{tabular}{p{.4\textwidth}|p{.5\textwidth}}
\textsf{rozpakowanie} & \textsf{wynik} \\\hline
\verb|a, b = [1, 2]| & \verb|a = 1, b = 2|\\
\verb|a, b, c = [1, 2, 3]| & \verb|a = 1, b = 2, c = 3|\\
\verb|a, b, c, d = [1, 2, 3, 4]| & \verb|a = 1, b = 2, c = 3, d = 4| \\
\verb|a, b, c, d = 'kura'| & \verb|a = 'k', b = 'u', c = 'r', d = 'a'| \\
\verb|a, b, c, d = 'kot'| & błąd -- za mało wartości do rozpakowania \\

\end{tabular}

\textbf{\arabic{zadanie}.}\addtocounter{zadanie}{1}

\begin{tabular}{p{.4\textwidth}|p{.5\textwidth}}
\textsf{rozpakowanie} & \textsf{wynik} \\\hline
\verb|a, *b = [1, 2, 3, 4]| & \verb|a = 1, b = [2, 3, 4]| \\
\verb|*a, b = [1, 2, 3, 4]| & \verb|a = [1, 2, 3], b = 4| \\
\verb|a, *b, c = [1, 2, 3]| & \verb|a = 1, b = [2], c = 3| \\
\verb|a, *b = 'kucyk'| & \verb|a = 'k', b = ['u', 'c', 'y', 'k']| \\
\verb|*a, b, c = 'kucyk'| & \verb|a = ['k', 'u', 'c'], b = 'y', c = 'k'| \\
\verb|a, *b, c = 'kucyk'| & \verb|a = 'k', b = ['u', 'c', 'y'], c = 'k'| \\
\verb|a, *b = 'kucyk'| & \verb|a = 'k', b = ['u', 'c', 'y', 'k']|\\
\verb|*a, b = 'kucyk'| & \verb|a = ['k', 'u', 'c', 'y'], b = 'k'|\\
\end{tabular}

\textbf{\arabic{zadanie}.}\addtocounter{zadanie}{1}

\begin{tabular}{p{.4\textwidth}|p{.5\textwidth}}
\textsf{rozpakowanie} & \textsf{wynik} \\\hline
\verb|*a, b, c = [1, 2, 3, 4]| & \verb|a = [1, 2], b = 3, c = 4| \\
\verb|*a, b = 'maska'| & \verb|a = ['m', 'a', 's', 'k'], b = 'a'| \\
\verb|*a, b, c = 'maska'| & \verb|a = ['m', 'a', 's'], b = 'k', c = 'a'| \\
\verb|a, *b, c = 'maska'| & \verb|a = 'm', b = ['a', 's', 'k'], c = 'a'| \\
\verb|a, b, *c = 'maska'| & \verb|a = 'm', b = 'a', c = ['s', 'k', 'a']| \\
\verb|*a, b = [1, 2, [3, 4]]| & \verb|a = [1, 2], b = [3, 4]| \\
\end{tabular}

KONIEC
\section{Wyrażenie warunkowe}

\textbf{\arabic{zadanie}.}\addtocounter{zadanie}{1} Wyznacz wartość wyrażeń warunkowych w poniższej tabeli.

\begin{tabular}{l|l|c}
\textsf{zmienna} & \textsf{wyrażenie warunkowe} & \textsf{wartość} \\\hline
\verb|s = 0b10| & \verb|3 if s == 2 else 0| & 3\\
\verb|s = 0o14| & \verb|s if s >= 12 else 0| &\\
\verb|s = 1| & \verb|1 if s else 0| &\\
\verb|s = 'a_0'| & \verb|1 if s else 0|&\\
\verb|s = 'a_0'| & \verb|1 if len(s) <= 3 else 0|&\\
\verb|s = 'a_0'| & \verb|'x' if s[0] == 'a' else 'y'|&\\
\verb|s = 'a_0'| & \verb|'x' if s[-1] != '2' else 'y'|&\\
\verb|s = 'a_0'| & \verb|'x' if s[-1] == '0' else 'y'|&\\
\verb|s = 'a_0'| & \verb|'x' if s[:0] else 'y'|&\\
\verb|s = ''| & \verb|3 if s else 9|&\\
\verb|s = []| & \verb|0 if s else 1|&\\
\verb|s = {}| & \verb|0 if s else 1|&\\
\end{tabular}
\newline


\textbf{\arabic{zadanie}.}\addtocounter{zadanie}{1} Niech \verb|z = 'alf'|. Oblicz samodzielnie wartość wyrażeń.

\begin{multicols}{2}
\begin{enumerate}[label=\arabic*.]
    \item \verb|len(z) >= 3|
    \item \verb|len(z) < 4|
    \item \verb|len(z) >= 1|
    \item \verb|'l' in z|
    \item \verb|'fl' in z|
    \item \verb|'af' in z|
    \item \verb|'alf' in z|
    \item \verb|'lf' in z|
    \item \verb|1 if len(z) == 3 else -1|
    \item \verb|2 if len(z) <= 3 else 0|
    \item \verb|2 if len(z) % 2 == 1 else 0|

\end{enumerate}
\end{multicols}


\section{Listy}

\textbf{\arabic{zadanie}.}\addtocounter{zadanie}{1} Utwórz szybko listę złożoną:
\begin{enumerate}[label=\arabic*.]
    \item ze stu liczb równych \verb|0|
    \item z tysiąca liczb równych \verb|1|
    \item z pięciuset liter \verb|'a'|
\end{enumerate}


\textbf{\arabic{zadanie}.}\addtocounter{zadanie}{1} Wykonaj poniższe polecenia.
\begin{enumerate}[label=\arabic*.]
    \item Napisz wyrażenie, które tworzy listę złożoną z liczb \verb|1, 2, 3|.
    %~ [1, 2, 3]
    \item Napisz inne wyrażenie, które tworzy taką samą listę.
    %~ list(range(1, 4))
    %~ [j for j in range(1, 4)]
    \item Przypisz tę listę zmiennej \verb|l|.
    \item Napisz pięć różnych wyrażeń, które z pomocą listy \verb|l| tworzą listę \verb|[1, 2, 3, 4]|.
    %~ l.append(4) // l.extend([4]) // l += [4] // l = [*l, 4] // l[-1:] = [3, 4]
    \item Napisz dwa wyrażenia, które z listy \verb|l| usuwają ostatni element.
    %~ l.pop(), del l[-1]
\end{enumerate}

\textbf{\arabic{zadanie}.}\addtocounter{zadanie}{1} Wykonaj sekwencję poleceń, używając odpowiednich konstrukcji Pythona, funkcji i metod.

\begin{enumerate}[label=\arabic*.]
    \item Utwórz listę \verb|l=[5, 5, 5]|.
    \item Na końcu listy \verb|l| dodaj kolejno elementy \verb|7, 8, 9|.
    \item Między elementy \verb|5| i \verb|7| wstaw \verb|6|.
    \item Zerowy element listy \verb|l| zamień na \verb|3|, pierwszy na \verb|4|
    \item Usuń ostatni element listy \verb|l|.
    \item Rozszerz listę \verb|l| o listę \verb|[10, 11, 12]|.
    \item Usuń trzeci (tj. mający indeks 2) element listy \verb|l|.
\end{enumerate}


\textbf{\arabic{zadanie}.}\addtocounter{zadanie}{1} Dana jest lista \verb|l = [1, 2, 3, 4, ..., 20]|. Używając notacji ,,slice'' \verb|[i:j:k]| skopiuj z niej poniższe listy:

\begin{multicols}{2}
\begin{enumerate}[label=\arabic*.]
    \item \verb|[1, 2, 3, ..., 10]|
    \item \verb|[10, 11, 12, ..., 20]|
    \item \verb|[2, 4, 6, 8, 10]|
    \item \verb|[10, 12, 14, 16, 18, 20]|
    \item \verb|[1, 3, 5, 7, 9]|
    \item \verb|[11, 13, 15, 17, 19]|
    \item \verb|[1, 3, 5, ..., 19]|
    \item \verb|[2, 4, 6, ..., 20]|
    \item \verb|[1, 4, 7, ..., 19]|
    \item \verb|[2, 5, 8, ..., 20]|
\end{enumerate}

\end{multicols}


\textbf{\arabic{zadanie}.}\addtocounter{zadanie}{1} Używając ,,slice'' zmodyfikuj listę \verb|l| z poprzedniego zadania, jak poniżej.

\begin{enumerate}[label=\arabic*.]
    \item \verb|[0, 0, ..., 0, 10, 11, ..., 20]|
    \item \verb|[1, 2, ..., 9, 10, 10, ..., 10]|
    \item \verb|[1, 2, 19, 20]|
    \item \verb|[1, 2, 3, 18, 19, 20]|
    \item \verb|[1, 2, 3]|
    \item \verb|[1, 2, 19, 20]|
    \item \verb|[1, 2, ..., 7, 8, 8, 8, 8, 12, 13, ..., 20]|
    \item \verb|[1, 2, ..., 9, 10, 10, ..., 10]|
    \item \verb|[7, 2, 7, 4, 7, 6, 7, 8, 7, 10, 7, 12, 7, 14, 7, 16, 7, 18, 7, 20]|

\end{enumerate}

\textbf{\arabic{zadanie}.}\addtocounter{zadanie}{1} Dana jest lista \verb|l| taka, jak w poprzednim zadaniu. Napisz wyrażenia, które korzystają z \verb|l| i dają w wyniku poniższe listy.

\begin{multicols}{2}
\begin{enumerate}[label=\arabic*.]
    \item \verb |[1, 3, 5, 7, 9]|
    \item \verb |[1, 2, 3, 4, 5]|
    \item \verb |[2, 4, 6, 8, 10]|
    \item \verb |[1, 4, 7, 10]|
    \item \verb |[2, 6, 10]|
    \item \verb |[2, 5, 8]|
    \item \verb |[3, 6, 9]|
    \item \verb |[2, 6]| (dwoma sposobami)
    \item \verb |[5, 8]| (dwoma sposobami)
    \item \verb |[3, 9]| (dwoma sposobami)
    \item \verb |[3, 4, 5, 6]|
    \item \verb |[15, 16, 17, 18, 19, 20]|
\end{enumerate}
\end{multicols}


\section{List comprehension}
List comprehension to bardzo wygodna notacja, którą można stosować nie tylko w odniesieniu do list.

\important{Ogólna składnia \emph{list comprehension} jest następująca:

\vspace{.5em}
\centerline{\texttt{[ ... for x in} \emph{ob\_iterowalny} \texttt{if war\_logiczny]}}
\vspace{.5em}
Powyższy wariant tworzy listę -- użycie nawiasu klamrowego utworzy zbiór lub słownik, w zależności od szczegółów konstrukcji.
\vspace{.5em}

\texttt{...} należy rozumieć jako realizację pewnej operacji.
}

\paragraph{Przykład} Niech \verb|iterable = 'Python'|.
\begin{enumerate}[]
    \item \verb|[ord(x) for x in 'Python']| $\rightarrow$ \verb|[80, 121, 116, 104, 111, 110]|
    \item \verb|[i for i, c in enumerate('Python')]| $\rightarrow$ \verb|[0, 1, 2, 3, 4, 5]|
    \item \verb|{i: c for i, c in enumerate('Python')}| $\rightarrow$\newline
    \verb|{0: 'P', 1: 'y', 2: 't', 3: 'h', 4: 'o', 5: 'n'}|
    \item \verb|{c for c in 'Python' if c != 'o'}| $\rightarrow$ \verb|{'P', 'y', 't', 'h', 'n'}| (to jest zbiór, zatem w rzeczywistości kolejność liter może być różna)
    \item \verb|[c for i, c in enumerate('Python') if i % 2]| $\rightarrow$ \verb|['y', 'h', 'n']|
\end{enumerate}

\textbf{\arabic{zadanie}.}\addtocounter{zadanie}{1} Dane są zmienne
\begin{enumerate}[]
    \item \verb|a = 'kosmodrom'|
    \item \verb|b = 'alfabetyzacja'|
    \item \verb|c = 'luminescencja'|
    \item \verb|x = [1, 2, 3, 4, 5, 6, 7, 8, 9, 10]|
\end{enumerate}

Korzystając z tych zmiennych napisz list comprehension, którego wynikiem jest:
\begin{enumerate}[label=\arabic*.]
    \item \verb|['k', 'o', 's', 'm', 'o', 'd', 'r', 'o', 'm']|
    \item \verb|['a', 'l', 'f', 'a', 'b', 'e', 't', 'y', 'z', 'a', 'c', 'j', 'a']|
    \item \verb|['l', 'u', 'm', 'i', 'n', 'e', 's', 'c', 'e', 'n', 'c', 'j', 'a']|
    \item \verb|['l', 'u', 'm', 'i', 'n', 'e', 's', 'e', 'n', 'j', 'a']|
    \item \verb|['l', 'm', 'n', 's', 'c', 'n', 'c', 'j']|
    \item \verb|[2, 4, 6, 8, 10, 12, 14, 16, 18, 20]|
    \item \verb|[1, 4, 9, 16, 25, 36, 49, 64, 81, 100]|
    \item \verb|[1, 3, 5, 7, 9]|
    \item \verb|[1, 2, 3, 4, 5]|
    \item \verb|[2, 4, 6, 8, 10]|
    \item \verb|[15, 16, 17, 18, 19, 20]|
    \item \verb|['a', 'a', 'a', 'a']|
    \item \verb|['a', 'a', 'b', 'a', 'c', 'a']|
    \item \verb|['o', 'o', 'o']|
    \item \verb|[97, 108, 102, 97, 98, 101, 116, 121, 122, 97, 99, 106, 97]|
\end{enumerate}


\section{Teksty}
\textbf{\arabic{zadanie}.}\addtocounter{zadanie}{1} Za pomocą operatora \verb|*| utwórz tekst złożony z:
\begin{enumerate}[label=\arabic*.]
    \item ze stu liter \verb|'a'|
    \item czterdziestu myślników \verb|'-'|
    \item trzydziestu gwiazdek \verb|'*'|
\end{enumerate}


\textbf{\arabic{zadanie}.}\addtocounter{zadanie}{1} Za pomocą operatora \verb|+| utwórz z tekstów \verb|'Krowa'|, \verb|'daje'|, \verb|'mleko'| teksty
\begin{enumerate}[label=\arabic*.]
    \item \verb|'Krowa daje mleko'|
    \item \verb|'Krowa mleko daje'|
    \item \verb|'mleko daje Krowa'|
\end{enumerate}


\textbf{\arabic{zadanie}.}\addtocounter{zadanie}{1} Z trzech tekstów \verb|'Trzeba'|, \verb|'więcej'|, \verb|'pracować'| utwórz teksty
\begin{enumerate}[label=\arabic*.]
    \item \verb|'Trzeba więcej pracować'|
    \item \verb|'więcej Trzeba pracować'|
    \item \verb|'pracować Trzeba więcej'|
    \item \verb|'więcej pracować Trzeba'|
    \item \verb|'Trzeba pracować więcej'|
    \item \verb|'pracować więcej Trzeba'|
\end{enumerate}

\subsection{Metody obiektu \texttt{str}}

\textbf{\arabic{zadanie}.}\addtocounter{zadanie}{1} Z trzech tekstów \verb|'Trzeba'|, \verb|'więcej'|, \verb|'pracować'| utwórz poniższe teksty. Wykorzystaj metodę \verb|capitalize()|.
\begin{enumerate}[label=\arabic*.]
    \item \verb|'Trzeba Więcej Pracować'|
    \item \verb|'Więcej Trzeba Pracować'|
    \item \verb|'Pracować Trzeba Więcej'|
    \item \verb|'więcej Pracować Trzeba'|
    \item \verb|'Trzeba Pracować Więcej'|
    \item \verb|'Pracować Więcej Trzeba'|
\end{enumerate}


\textbf{\arabic{zadanie}.}\addtocounter{zadanie}{1} Utwórz zmienną \verb|motto| i nadaj jej wartość \verb|'Nobody is gonna turn Me round'|. Użyj odpowiedniej metody, by uzyskać tekst:

\begin{enumerate}[label=\arabic*.]
    \item 'NOBODY IS GONNA TURN ME ROUND'
    \item 'nobody is gonna turn me round'
    \item 'Nobody is gonna turn me round'
    \item 'Nobody Is Gonna Turn Me Round'
\end{enumerate}

\textbf{\arabic{zadanie}.}\addtocounter{zadanie}{1} Korzystając z tekstu \verb|'nie lubię myć nóg'| i wskazanych metod wykonaj poniższe polecenia:

\begin{enumerate}[label=\arabic*.]
    \item \verb|split()| -- utwórz zmienną \verb|l| i nadaj jej wartość \verb|['nie', 'lubię', 'myć', 'nóg']|
    \item \verb|join()| -- zamień element \verb|'nie'| na \verb|'Nie'|, a następnie zmiennej \verb|ll| nadaj wartość \verb|'Nie-lubię-myć-nóg'|
    \item \verb|count()| -- podaj, ile razy występuje w tym tekście \verb|'i'|
    \item \verb|rindex()| -- jaki jest indeks ostatniej litery \verb|'i'| w tym tekście
    \item \verb|find()| -- podaj indeks, pod jakim w danym tekście występuje tekst \verb|'my'|
    \item \verb|split()| -- teraz nadaj zmiennej \verb|l| wartość \verb|['nie', 'lubię', 'myć', 'nóg']|
    \item \verb|join()| -- zmiennej \verb|l| nadaj wartość \verb|'Nie ** lubię ** myć ** nóg'|
    \item \verb|replace()| -- zmiennej \verb|l| nadaj teraz wartość \verb|'Nie -- lubię -- myć -- nóg'|
    \item \verb|startswith()| -- sprawdź, czy tekst \verb|l| zaczyna się od \verb|'Nie'|
    \item \verb|islower()| -- sprawdź, czy tekst \verb|l| składa się wyłącznie z małych liter

\end{enumerate}


\textbf{\arabic{zadanie}.}\addtocounter{zadanie}{1} Dane są zmienne
\begin{enumerate}[]
    \item \verb|a = 'elektroluminescencja'|,
    \item \verb|b = 'alfabetyzaja'|,
    \item \verb|c = 'kosmodrom'|.
\end{enumerate}

Napisz wyrażenie, którego wynikiem jest tekst
\begin{enumerate}[label=\arabic*.]
    \item \verb|'alfabetyzacjakosmodrom'|
    \item \verb|'alf'|
    \item \verb|'kos'|
    \item \verb|'elektro'|
    \item \verb|'alfabet'|
    \item \verb|'luminescencja'|
    \item \verb|'m'| - ostatni znak tekstu c
    \item \verb|'om'|
    \item \verb|'k'| - pierwszy znak tekstu c
    \item \verb|'osm'|
    \item \verb|'trol'|
    \item \verb|'mordomsok'|
    \item \verb|lista ['k', 'o', 's', 'm', 'o', 'd', 'r', 'o', 'm']|
    \item \verb|'ajcazytebafla'|

    \item \verb|'***kosmodrom***'|
    \item \verb|'Kosmodrom'|
    \item \verb|'KOSMODROM'|


\end{enumerate}

\subsubsection{Metoda \texttt{format()}}

\section{Formatowanie tekstów}

\textbf{\arabic{zadanie}.}\addtocounter{zadanie}{1} Dane są zmienne:
\begin{enumerate}[ ]
    \item \verb|e = 3.21|
    \item \verb|f = 16/7|
    %~ \item \verb|i = 4|
    \item \verb|j = 7|
    \item \verb|str = 'szpak'|
    \item \verb|str_2 = 'kos'|
\end{enumerate}

Wypisz wartości wskazanych zmiennych, korzystając z funkcji \verb|format()|. Użyj podanego tekstu jako szablonu.

TO DO -------------------- FLOAT FLOAT {:0.3f}

\begin{enumerate}[label=\arabic*.]
    \item \verb|e, f, j| -- \verb|'{} {} {}'|
    \item \verb|e, f, j| -- \verb|'e={} f={} j={}'|

    \item \verb|e, f, j| -- \verb|'{0}, {1}, {2}'|
    \item \verb|e, f, j| -- \verb|'e={0}, f={1}, j={2}'|

    \item \verb|e, f, j| -- \verb|'f={1}, j={2}, e={0}'|
    \item \verb|e, f, j| -- \verb|'{0}-{0}-{2}'|
    \item \verb|e, f, j| -- \verb|'{2} / {1} / {0}'|
    \item \verb|e, f, j| -- \verb|'{1} : {0} : {2}'|

    \item \verb|e, f, j| -- \verb|'{:_^8}-{}-{:*^7}'|
    \item \verb|e, f, j| -- \verb-'{0:_<8} | {1} | {2:*^9}'-

    \item \verb|e, f, j| -- \verb|'e={: >8}, f={:.3}-{:*^7}'|
    \item \verb|e, f, j| -- \verb-'e={0:_<8} | f={1:+^10.4} | {2:*^9}'-

    \item \verb|f| -- \verb|'f={1:+^10.1}|
    \item \verb|f| -- \verb|'f={1:+<10.2}|
    \item \verb|f| -- \verb|'f={1:+<10.3}|
    \item \verb|f| -- \verb|'f={1:+>10.4}|
    \item \verb|f| -- \verb|'f={1:+>10.5}|
    \item \verb|f| -- \verb|'f={1:+>10.6}|

    \item \verb|str, str_2| -- \verb|'{} i {}'|
    \item \verb|str, str_2| -- \verb|'{0} i {1}'|
    \item \verb|str, str_2| -- \verb|'{1} i {0}'|
    \item \verb|str, str_2| -- \verb|'{1} i {2}'|
    \item \verb|str, str_2| -- \verb|'{} i {}'|

\end{enumerate}

\textbf{\arabic{zadanie}.}\addtocounter{zadanie}{1} Korzystając ze zmiennych z poprzedniego zadania zbuduj szablon dla funkcji \verb|format()| i przekaż jej odpowiednie parametry, by uzyskać wskazany wynik.

\begin{enumerate}[label=\arabic*.]
    \item \verb|'3.21 + 2.2857142857142856 = 5.4957142857142856'|
    \item \verb|'3.21 + 2.29 = 5.50'|
    \item \verb|'szpak i szpak'|
    \item \verb|'szpak i SZPAK'|
    \item \verb|'7, 3, 1'|

\end{enumerate}

\textbf{\arabic{zadanie}.}\addtocounter{zadanie}{1} Wykonaj poprzednie zadanie z wykorzystaniem f-stringów zamiast \verb|format()|.



\textbf{\arabic{zadanie}.}\addtocounter{zadanie}{1}
Za pomocą metody \verb|format()| i odpowiedniego tekstu-szablonu uzyskaj identyczne efekty, jak w poprzednim zadaniu.


\section{Konwersja na różne układy}

\textbf{\arabic{zadanie}.}\addtocounter{zadanie}{1} Wykorzystaj funkcję \verb|int| do zamiany liczby z danego układu na dziesiętny (pamiętaj, by liczbę podać jako string).

\begin{multicols}{2}
\begin{enumerate}[label=\arabic*.]
    \item $3_{3}\rightarrow \dots$
    \item $11_{3}\rightarrow \dots$
    \item $12_{3}\rightarrow \dots$
    \item $22_{3}\rightarrow \dots$

    \item $10_{2}\rightarrow \dots$
    \item $11_{2}\rightarrow \dots$
    \item $111_{2}\rightarrow \dots$
    \item $101_{2}\rightarrow \dots$
    \item $1111_{2}\rightarrow \dots$
    \item $1000_{2}\rightarrow \dots$
    \item $10000_{2}\rightarrow \dots$

    \item $10_{8}\rightarrow \dots$
    \item $11_{8}\rightarrow \dots$
    \item $17_{8}\rightarrow \dots$
    \item $77_{8}\rightarrow \dots$
    \item $22_{8}\rightarrow \dots$

    \item $\mathrm{A_{16}}\rightarrow \dots$
    \item $\mathrm{B_{16}}\rightarrow \dots$
    \item $\mathrm{F_{16}}\rightarrow \dots$
    \item $\mathrm{FF_{16}}\rightarrow \dots$

\end{enumerate}
\end{multicols}

\textbf{\arabic{zadanie}.}\addtocounter{zadanie}{1} Sprawdź, czy prawdziwe są równości. Tam gdzie można, wpisz liczbę bezpośrednio w podanym systemie liczbowym.

\begin{multicols}{2}
\begin{enumerate}[label=\arabic*.]
    \item $1010_{2} = 10$
    \item $11_{2} = 10_{3}$
    \item $11_{16} = 21_{8}$
    \item $\mathrm{A_{16}} = 10$
    \item $\mathrm{A_{16}} = 12_{8}$

    \item $11_{4} = 4$
    \item $11_{4} = 5$
    \item $4_{5} = 4$
    \item $4_{5} = 100_{2}$

    \item $\mathrm{FA_{16}} = 235$
    \item $\mathrm{AA_{16}} = 170$
    \item $\mathrm{E_{16}} = 86_{8}$

\end{enumerate}
\end{multicols}

\textbf{\arabic{zadanie}.}\addtocounter{zadanie}{1} Podaj reprezentacje danych liczb w układach szesnastkowym, ósemkowym i dwójkowym:

\begin{tabular}{c | c | c | c}
\textsf{dec} & \textsf{hex} & \textsf{oct} & \textsf{bin}\\\hline
12  & C & 14 & 1100 \\\hline
    & F &    &  \\\hline
51  &   &    &  \\\hline
    &   & 111 & \\\hline
    & EE &  &   \\\hline
    &   &   & 111111 \\\hline
256 &   &   &   \\\hline
    &   & 1234 &    \\\hline
    & A1A &     &   \\

\end{tabular}


\section{Prosty import}

\textbf{\arabic{zadanie}.}\addtocounter{zadanie}{1}
Import:
\lstset{numbers=none}
\begin{lstlisting}
from math import sqrt, floor, ceil, log, log2, log10, pi
\end{lstlisting}
Przykładowe obliczenia:

\begin{multicols}{3}
    \begin{enumerate}[label=\arabic*.]
        \item \lstinline|print(sqrt(9))|
        \item \lstinline|print(ceil(sqrt(8)))|
        \item \lstinline|print(log2(8)|
        \item $\log_{2} 8$
        \item $\lceil{-2.7}\rceil$
        \item $\log_{10} 0{,}1$
        \item $\lfloor{12/3}\rfloor$
        \item $\lceil{12/3}$
        \item $\log_{2} 4$
        \item $\sqrt{16}$
        \item $\log_{2} 16$
        \item $\lfloor{1}\rfloor$
        \item $\lfloor{-2.7}\rfloor$
        \item $\lfloor{12/3}\rfloor$
        \item $\log_{10} 100$
        \item $\log_{2} 16$
        \item $\lceil{12/3}\rceil$
        \item $\lceil{\pi}\rceil$
        \item $\log_{10} 10 $
        \item $\log_{2} 128$
        \item $\lfloor{\pi}\rfloor$
        \item $\log_{10} 1$
    \end{enumerate}
\end{multicols}


\subsubsection{Instrukcja warunkowa}


Wzorując się na tym, uprość warunki w następujących instrukcjach:

%~ \begin{enumerate}[label=\arabic*.]
    %~ \item \verb|if x == True:|
    %~ \item \verb|if x != 0:|
    %~ \item \verb|if x > 0 and x > 1:|
    %~ \item \verb|if x < 0 or x >0:|
    %~ \item \verb|if len(x) > 0:|
    %~ \item \verb|if len(x) != 0:|
    %~ \item \verb|if x != '':|
    %~ \item \verb|if x != []:|
    %~ \item \verb|if not x == '':|
    %~ \item \verb|if len(x) > 0 and y in x:|

%~ \end{enumerate}




%~ TO DO ================

%~ warto przeczytać również to:

%~ * https://docs.python.org/3/library/stdtypes.html
%~ * https://docs.python.org/3/library/itertools.html
%~ * https://docs.python.org/3/library/array.html
%~ * https://docs.python.org/3/library/bisect.html
%~ * https://docs.python.org/3/library/collections.html

%~ to pierwsze opis standardowych typów (listy, stringi, słowniki...) i są tu wymienione chyba wszystkie metody stringowe, kolejne strony to przydatne biblioteki.

%~ warto byłoby większość przykładów ilustrujących przećwiczyć samodzielnie.

%~ 4 sposoby na utworzenie listy liczb od 1 do 10

%~ 1) list(range(1, 10))
%~ 2) [j for j in range(1, 10)]
%~ 3) list(j for j in range(10))
%~ 4) [*range(1, 10)]

%~ 2) i 3) to nie jest to samo. 4) jest fajny i naucz się tego: gwiazdka * służy do rozpakowywania "obiektów listopodobnych" (list, stringów i nie tylko, ale o tym kiedy indziej).

%~ cztery sposoby na zamianę stringu 'wercyngetoryks' na listę:

%~ 1) list('wercyngetoryks')
%~ 2) [c for c in 'wercyngetoryks']
%~ 3) list(c for c in 'wercyngetoryks')
%~ 4) [*'wercyngetoryks']

%~ również tutaj 2) i 3) to nie to samo i nie jest to sztuczne

%~ jak z dwóch list zrobić jedną? niech a = [1, 2, 3], b = [4, 5, 6]

%~ 1) a.extend(b)
%~ 2) a + b
%~ 3) [*a, *b]
%~ 4) a[-1:] = b

%~ są tu pewne subtelności - w 1) lista a jest rozszerzana o elementy b - a po prostu się zmienia. w 2) i 3) produkowana jest nowa lista. w 4) lista a znów jest rozszerzana o b - CZY ROZUMIESZ jak to działa?


\section{Pętle \texttt{for} i \texttt{while}}
\textbf{\arabic{zadanie}.}\addtocounter{zadanie}{1} Napisz pętlę \verb|for|, która z odpowiednio dobranym obiektem \verb|range()| wypisuje:

\begin{enumerate}[label=\arabic*.]
    \lstset{numbers=none}
    \item \begin{lstlisting}
    for j in range(1, 21):
        print(j)
    \end{lstlisting}

    \item \begin{lstlisting}
    for j in range(-5, 11):
        print(j)
    \end{lstlisting}

    \item \begin{lstlisting}
    for j in range(-15, 9):
        print(j)
    \end{lstlisting}

    \item \begin{lstlisting}
    for j in range(0, 17, 2):
        print(j)
    \end{lstlisting}

    \item \begin{lstlisting}
    for j in range(0, -17, -2):
        print(j)
    \end{lstlisting}

    \item \begin{lstlisting}
    for j in range(-3, 22, 3):
        print(j)
    \end{lstlisting}

    \item \begin{lstlisting}
    for j in range(-3, 26, 4):
        print(j)
    \end{lstlisting}

    \item \begin{lstlisting}
    for j in range(-7, 20, 2):
        print(j)
    \end{lstlisting}

    \item \begin{lstlisting}
    for j in range(7, 48, 5):
        print(j)
    \end{lstlisting}

    \item \begin{lstlisting}
    for j in range(20, -12, -3):
        print(j)
    \end{lstlisting}
\end{enumerate}

\textbf{\arabic{zadanie}.}\addtocounter{zadanie}{1}
\lstset{numbers=none}
\begin{enumerate}[label=\arabic*.]
    \item \begin{lstlisting}
    for j in range(1, 32, 2):
        print(j)
    \end{lstlisting}
    Inny wariant:
    \begin{lstlisting}
    for j in range(16):
        print(2*j + 1)
    \end{lstlisting}

    \item \begin{lstlisting}
    for j in range(11):
        print(j ** 2)
    \end{lstlisting}

    \item \begin{lstlisting}
    for j in range(4, 17):
        print(j ** 0.5)
    \end{lstlisting}

    \item \begin{lstlisting}
    for j in range(1, 16):
        print((j ** 2 + 1) % 13)
    \end{lstlisting}

    \item \begin{lstlisting}
    for j in range(2, 101):
        print(3 ** (1 / j))
    \end{lstlisting}
\end{enumerate}

\textbf{\arabic{zadanie}.}\addtocounter{zadanie}{1} \texttt{l = [8, 14, 20, 7, 1, 8, 14, 12, 7, 14, 15, 14, 5, 11, 4]}
\begin{enumerate}[label=\arabic*.]
    \lstset{numbers=none}
    \item \begin{lstlisting}
    mx = l[0]
    for x in l[1:]:
        if x > mx:
            mx = x
    print(mx)
    \end{lstlisting}

    \item \begin{lstlisting}
    mx = l[0]
    for x in l[1:]:
        if x < mx:
            mx = x
    print(mx)
    \end{lstlisting}

    \item \begin{lstlisting}
    s = 0
    for x in l:
        s += x
    print(s)
    \end{lstlisting}

    \item \begin{lstlisting}
    s = 0
    for x in l:
        s += x
    print(s / len(l))
    \end{lstlisting}
    Bezpieczniejsza wersja, uwzględniająca przypadek \lstinline|l = []|, byłaby taka:

    \begin{lstlisting}
    s = 0
    for x in l:
        s += x
    print(s / (len(l) or 1)
    # inny wariant powyższego: print(s / len(l) if l else 1)
    \end{lstlisting}

    \item \begin{lstlisting}
    mx, dx = l[0], l[0] - 1
    for x in l[1:]:
        if x > mx:
            dx, mx = mx, x
    print(dx)
    \end{lstlisting}

    \item \begin{lstlisting}
    licznik = 0
    for x in l:
        if x == 14:
            licznik += 1
    print(licznik)
    \end{lstlisting}

\end{enumerate}

%~ \textbf{\arabic{zadanie}.}\addtocounter{zadanie}{1} Rozwiąż zadania \thewhileloop-\thewhileloopx\ za pomocą pętli \texttt{while}. Oczywiście, nie wolno korzystać z \verb|range()|.

\section{Pliki tekstowe}
\textbf{\arabic{zadanie}.}\addtocounter{zadanie}{1} Wariant czwarty jest najmniej zalecany.

\begin{enumerate}[label=\arabic*.]
    \lstset{numbers=none}
    \item \begin{lstlisting}
    with open('dane.txt') as plik:
        p = plik.readlines()
    \end{lstlisting}

    \item \begin{lstlisting}
    with open('dane.txt') as plik:
        p = list(plik)
    \end{lstlisting}

    \item \begin{lstlisting}
    with open('dane.txt') as plik:
        p = [l for l in plik]
    \end{lstlisting}

    \item \begin{lstlisting}
    p = []
    with open('dane.txt') as plik:
        for l in plik:
            p.append(l)
    \end{lstlisting}

\end{enumerate}

\textbf{\arabic{zadanie}.}\addtocounter{zadanie}{1}
\begin{enumerate}[label=\arabic*.]
    \lstset{numbers=none}
    \item \begin{lstlisting}
    with open('dane.txt') as plik:
        print(plik.readline())
    \end{lstlisting}

    \item \begin{lstlisting}
    with open('dane.txt') as plik:
        print(plik.readline())
        print(plik.readline())
    \end{lstlisting}

    \item \begin{lstlisting}
    licznik = 0
    with open('dane.txt') as plik:
        for l in plik:
            licznik += 1
    print(licznik)
    \end{lstlisting}

    \item \begin{lstlisting}
    dl, w = 0, ''
    with open('dane.txt') as plik:
        for l in plik:
            ll = l.strip()
            if len(ll) > dl:
                w = ll
                dl = len(ll)
    print(dl)
    \end{lstlisting}

    \item \begin{lstlisting}
    with open('dane.txt') as plik:
        w = plik.readline().strip()
        dl = len(w)
        for l in plik:
            ll = l.strip()
            if len(ll) < dl:
                w = ll
                dl = len(ll)
    print(dl)
    \end{lstlisting}

    \item \begin{lstlisting}
    licznik = 0
    with open('dane.txt') as plik:
        for l in plik:
            if len(l.strip()) >= 40:
                licznik += 1
    print(dl)
    \end{lstlisting}

\end{enumerate}

\textbf{\arabic{zadanie}.}\addtocounter{zadanie}{1}
    \lstset{numbers=none}
    \begin{lstlisting}
    with open('witaj.txt', 'r') as plik:
        plik.write('Panie Janie, niech pan wstanie!\n')
    \end{lstlisting}

\textbf{\arabic{zadanie}.}\addtocounter{zadanie}{1}
   \lstset{numbers=none}
   \begin{lstlisting}
    with open('witaj.txt', 'a') as plik:
        plik.write('Panie Janie, rano wstań!\n')
    \end{lstlisting}

\textbf{\arabic{zadanie}.}\addtocounter{zadanie}{1}
   \lstset{numbers=none}
   \begin{lstlisting}
    with open('iloczyny.txt', 'w') as plik:
        for j in range(1, 10):
            plik.write(f'{1 * j}\n')
    \end{lstlisting}

\textbf{\arabic{zadanie}.}\addtocounter{zadanie}{1}
   \lstset{numbers=none}
   \begin{lstlisting}
    with open('iloczyny.txt', 'w') as plik:
        for i in range(1, 10):
            for j in range(1, 10):
                plik.write(f'{i * j}\n')
    \end{lstlisting}

\textbf{\arabic{zadanie}.}\addtocounter{zadanie}{1}
   \lstset{numbers=none}
   \begin{lstlisting}
    with open('uczniowie.txt', 'w') as plik:
        plik.write('imię nazwisko klasa\n')
        plik.write('Anna Zielińska 3a\n')
        plik.write('Adam Pawłowski 3b\n')
        plik.write('Paulina Stefaniak 3d\n')
   \end{lstlisting}

\textbf{\arabic{zadanie}.}\addtocounter{zadanie}{1}
   \lstset{numbers=none}
   \begin{lstlisting}
    with open('gwiazdki.txt', 'w') as plik:
        for j in range(1, 101):
            plik.write(f'{j * "*"}\n')
   \end{lstlisting}

\end{document}
