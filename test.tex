\documentclass[a4paper]{article}
\usepackage[T1]{fontenc}
\usepackage[utf8]{inputenc}
\usepackage{lmodern}
\usepackage{enumerate}
\usepackage[margin=1in]{geometry}
\usepackage{listings}

\usepackage{polski}

\usepackage{tabularx}

\usepackage{color}
\definecolor{gray}{rgb}{0.8, 0.8, 0.8}

% makro nadające tło 'gray' tekstowi podanemu jako parametr - stara wersja
% \newcommand{\important}[1]{\colorbox{gray}{\begin{minipage}[t]{\linewidth}#1\end{minipage}}}

% \newcommand{\important}[1]{\colorbox{gray}{\parbox[t]{0.9\linewidth}{#1}}}

\newcommand{\important}[1]{
    \begin{center}\colorbox{gray}{
        \begin{minipage}[t]{0.9\textwidth}{#1}
        \end{minipage}
    }
    \end{center}
}

\newcommand{\kwords}[1]{\begin{center}\colorbox{gray}{\parbox{0.9\textwidth}{\textbf{Słowa kluczowe:} \emph{#1}}}\end{center}\vspace{1eM}}

\newcommand{\key}[1]{\textbf{#1}}


%~ \usepackage[polish]{babel}
%~ \selectlanguage{polish}

\setlength{\parindent}{0pt}
\setlength{\parskip}{1ex plus 0.5ex minus 0.2ex}

\title{Podstawy Pythona}
%~ \author{Wojciech Świderski}

\begin{document}

\subsubsection*{Matematyczne}
\begin{tabularx}{\textwidth}{lp{0.2\textwidth}|lp{0.6\textwidth}|lp{0.2\textwidth}}
abs() & wartość bezwzględna liczby & \verb|abs(-3)|$\ \to\ $\verb|3|\\
min() & najmniejsza wartość & \verb|min(3, 2, 5)|$\ \to\ $\verb|2| \\
max() & największa wartość & \verb|max([2, -1, 3])|$\ \to\ $\verb|3|\\
round() & zaokrągla liczbę do wskazanej dokładności &  \verb|round(0.267, 2)|$\ \to\ $\verb|0.27|\\

divmod() & zwraca wynik dzielenia całkowitoliczbowego i jego resztę & \verb|divmod(17, 4)|$\ \to\ $\verb|(4, 1)|\\
pow() & podnosi liczbę do wskazanej potęgi & \verb|pow(2, 3)|$\ \to\ $\verb|8|\\
sum() & sumę liczb z podanej listy & \verb|sum(1, 2, 3, 4)|$\ \to\ $\verb|10|\\
any() & True, gdy choćby jeden z warunków podanych na liście jest spełniony & \verb|any([0, False, 1, 1==2])|$\ \to\ $\verb|True|\\
all() & True, gdy wszystkie warunki podane na liście są spełnione & \verb|any([1, True, 2, 1<=1])|$\ \to\ $\verb|True|\\
filter() & iterator złożony z tych obiektów, które spełniają wskazany warunek & \verb|list(filter)|$\ \to\ $ \\
len(s) & długość tekstu & \verb|len("Python")|$\ \to\ $\verb|6|
\end{tabularx}

\begin{enumerate}[--]
    \item alfa
\end{enumerate}

\end{document}
