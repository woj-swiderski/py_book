\section{Dodatek B --- formatowanie f-stringów}
\label{dodatek-B}

Formalny opis języka formatowania można znaleźć na \href{https://docs.python.org/3/library/string.html#formatspec}{stosownej podstronie} witryny \verb|python.org| --- tutaj ograniczymy się do reprezentatywnych przykładów.

Przedstawiony schemat formatowania może być używany wewnątrz f-stringów oraz jako parametr metody \verb|format()| i funkcji \verb|format()| z \hyperref[funkcja-format]{biblioteki standardowej}.

Ogólny schemat formatowania wygląda następująco:

{\small
\verb|{parametr: [[wypełniacz]wyrównanie][znak][\#][0][szerokość][znacznik grupowania][.dokładność][typ]|
}

\texttt{
\begin{tabular}{ l l }
wypełniacz & dowolny znak\\
wyrównanie & <, >, =, \^{} \\
znak liczby & +, -, " " (spacja) \\
szerokość & liczba całkowita \\
separator tysięcy & \_, "," (przecinek) \\
dokładność & liczba całkowita \\
typ & b, c, d, e, E, f, F, g, G, n, o, s, x, X, \%
\end{tabular}
}

\subsubsection*{Elementarz}

\verb|"{0}, {1}, {2}".format("a", "b", "c")|$\ \to\ $\verb|'a, b, c'|

\verb|"{}, {}, {}".format("a", "b", "c")|$\ \to\ $\verb|'a, b, c'|

\verb|"{2}, {1}, {0}".format("a", "b", "c")|$\ \to\ $\verb|'c, b, a'|

Argumenty mogą się powtarzać i nie wszystkie muszą wystąpić:

\verb|"{0}, {1}, {0}".format("a", "b", "c")|$\ \to\ $\verb|'a, b, a'|

Argumenty funkcji \verb|format()| mogą mieć nazwy:

\verb|"Wzrost: {wzrost} cm, wiek: {wiek} lat".format(wzrost=184, wiek=45)"|

$\ \to\ $\verb|'Wzrost: 184 cm, wiek 45 lat'|

Jeżeli zdefiniujemy

\verb|dane = {"wiek": 45, "wzrost": 184}|,

to możemy napisać, wykorzystując operator \verb|**| rozpakowania

\verb|"Wzrost: {wzrost} cm, wiek: {wiek} lat".format(**dane)"|$\ \to\ $\verb|Wzrost: 184 cm, wiek 45 lat|

Można również odwoływać się do indeksów:

\verb|wektor = (-3, 4)|

\verb|"x: {0[0]}, y: {0[1]}".format(wektor)|$\ \to\ $\verb|'x: -4, y: 4'|

\subsubsection*{Wyrównywanie}

\verb|"{:_<20}".format("do lewej")|$\ \to\ $\verb|"do lewej_______"|

\verb|"{:.>20}".format("do prawej")|$\ \to\ $\verb|"......do prawej"|

\verb|"{:*^20}".format("do środka")|$\ \to\ $\verb|"*****do środka******"|

Tak z kolei używa się znaków \verb|+| i \verb|-|:

\verb|pi = 3.1416|

\verb|f"pi = {pi:*^+10f}, pi = {pi:-f}"|$\ \to\ $\verb|'pi = +3.141593*, pi = 3.141593'|

\verb|+| wymusza używanie znaku \emph{zawsze}; \verb|-| wyłącznie w przypadku liczb ujemnych; spacja natomiast sprawia, że liczby dodatnie poprzedzone będą spacją, a liczby ujemne minusem.

\subsubsection*{Format binarny, ósemkowy, szesnastkowy}

\verb|"int: {0:d};  hex: {0:x};  oct: {0:o};  bin: {0:b}".format(17)|$\ \to\ $

\verb|'int: 17;  hex: 11;  oct: 21;  bin: 10001'|

\subsubsection*{Separator tysięcy}

\verb|"{:,}".format(1234567890)|$\ \to\ $\verb|'1,234,567,890'|

