\documentclass[a4paper]{article}
\usepackage[T1]{fontenc}
\usepackage[utf8]{inputenc}
\usepackage{lmodern}
\usepackage{amssymb}

%~ \usepackage{enumerate}
\usepackage{enumitem}
%~ \usepackage[margin=1in]{geometry}
\usepackage{geometry}
\usepackage{multicol}

\usepackage{color}
\definecolor{gray}{rgb}{0.8, 0.8, 0.8}
\definecolor{deepgreen}{rgb}{0.156, 0.43, 0}

\usepackage{listings}

\lstset{
language=Python,
basicstyle=\ttfamily,
keywordstyle=\bfseries,
tabsize=4,
numbers=left,
numberblanklines=false,
showstringspaces=false,
keepspaces,
keywordstyle=\color{blue},
stringstyle=\color{deepgreen}
}

\lstset{
literate={ą}{{\c a}}1
{ć}{{\' c }}1
{ę}{{\c e }}1
{ł}{{\l}}1
{ń}{{\'n}}1
{ó}{{\'o}}1
{ś}{{\'s}}1
{ź}{{\'z}}1
{ż}{{\.z}}1
}

\usepackage{polski}

\usepackage{graphicx}
\usepackage{longtable}
\usepackage{hyperref}

% makro nadające tło 'gray' tekstowi podanemu jako parametr - stara wersja
% \newcommand{\important}[1]{\colorbox{gray}{\begin{minipage}[t]{\linewidth}#1\end{minipage}}}

% \newcommand{\important}[1]{\colorbox{gray}{\parbox[t]{0.9\linewidth}{#1}}}

\newcommand{\important}[1]{
    \begin{center}\colorbox{gray}{
        \begin{minipage}[t]{0.9\textwidth}{#1}
        \end{minipage}
    }
    \end{center}
}

\newcommand{\kwords}[1]{\begin{center}\colorbox{gray}{\parbox{0.9\textwidth}{\textbf{Słowa kluczowe:} \emph{#1}}}\end{center}\vspace{1eM}}

\newcommand{\key}[1]{\textbf{#1}}

%~ \usepackage[polish]{babel}
%~ \selectlanguage{polish}

\setlength{\parindent}{0pt}
\setlength{\parskip}{1ex plus 0.5ex minus 0.2ex}

\title{Pytania i zadania}
%~ \author{Wojciech Świderski}

\begin{document}
\maketitle
\tableofcontents

% --- czy counter MUSI być zdefiniowany tu, czy może być w preambule?
\newcounter{zadanie}
\setcounter{zadanie}{1}
%------------------------------

\section{Wejście / wyjście}
\textbf{\arabic{zadanie}.}\addtocounter{zadanie}{1} Za pomocą funkcji \verb|input()| wczytaj z klawiatury dane, przekształć je do wskazanego typu.

\begin{tabular}{l|l|l}
\textsf{rodzaj danych} & \textsf{zachęta} & \textsf{input} \\\hline
liczba całkowita & ,,długość w cm: '' & \verb|d = int(input('długość w cm: '))|\\\hline
liczba całkowita & ,,waga w kg: '' & \verb|d = int(input('waga w kg: '))|\\\hline
wartość logiczna & ,,tak / nie?'' & \verb|ans = bool(input('tak / nie? '))|\\\hline
tekst & ,,podaj imię'' & \dots \\\hline
liczba rzeczywista & ,,podaj szerokość: '' & \dots \\\hline
liczba całkowita & ,,liczba gości: '' & \dots \\\hline
tekst & ,,nazwisko'' & \dots \\\hline
liczba całkowita & ,,wiek w latach'' & \dots \\\hline
liczba całkowita & ,,liczba pokojów'' & \dots \\\hline
liczba rzeczywista & ,,długość w metrach'' & \dots \\\hline
wartość logiczna & ,,więcej? '' & \dots
\end{tabular}

\textbf{\arabic{zadanie}.}\addtocounter{zadanie}{1} Dane są zmienne
\begin{enumerate}[]
    \item \verb|e = 3.21|
    \item \verb|f = 16/7|
    \item \verb|str = 'szpak'|
    \item \verb|str_2 = 'kos'|
\end{enumerate}

Wypisz wskazane zmienne za pomocą funkcji \verb|print()|, korzystając ze wskazanego separatora. Wypróbuj też różne wartości parametru \verb|end|.
\begin{multicols}{2}
\begin{enumerate}[label=\arabic*.]
    \item \verb|e, f|, separator \verb|'/'|
    \item \verb|e, f|, separator \verb|' / '|
    \item \verb|e, f|, separator \verb|', '|
    \item \verb|e, f|, separator \verb|',\n'|
    \item \verb|str, str_2|, separator \verb|' i '|
    \item \verb|str, str_2|, separator \verb|', '|
    \item \verb|str, str_2|, separator \verb|'\n'|
    \item \verb|str, str_2|, separator \verb|'\t'|
\end{enumerate}
\end{multicols}


\section{Zmienne}
\textbf{\arabic{zadanie}.}\addtocounter{zadanie}{1} Które nazwy są poprawnymi nazwami zmiennych w Pythonie?

\begin{multicols}{3}
\begin{enumerate}[label=\arabic*.]
    \item \verb|x|
    \item \verb|i|
    \item \verb|a|
    \item \verb|ą|
    \item \verb|ab|
    \item \verb|abc|
    \item \verb|a-bc|
    \item \verb|b_ę_c|
    \item \verb|a_bc|
    \item \verb|gość|
    \item \verb|?|
    \item \verb|!|
    \item \verb|a?|
    \item \verb|2|
    \item \verb|2x|
    \item \verb|2-x|
    \item \verb|2x|
    \item \verb|x-2|
    \item \verb|x_2|
    \item \verb|_|
    \item \verb|x_|
    \item \verb|_x_|
    \item \verb|1_|
    \item \verb|_1|
    \item \verb|zmienna|
    \item \verb|Zmienna|
    \item \verb|Python|
    \item \verb|Python_3.10|
    \item \verb|Python_3_10|
    \item \verb|ZmieNNa|
    \item \verb|to zmienna|
    \item \verb|imie|
    \item \verb|Imie|
    \item \verb|imie_Nazwisko|
    \item \verb|imie nazwisko|
    \item \verb|imie-nazwisko|
    \item \verb|pięć|
    \item \verb|krófka|
    \item \verb|*|
    \item \verb|_*|
\end{enumerate}
\end{multicols}

\textbf{\arabic{zadanie}.}\addtocounter{zadanie}{1}
Utwórz w interpreterze następujące zmienne i nadaj im wskazane wartości (zmiennej po lewej stronie strzałki powinieneś nadać wartość z jej prawej strony):

\begin{enumerate}[label=\arabic*.]
    \item \verb|l| $\leftarrow$ \verb|4|
    \item \verb|i| $\leftarrow$ \verb|-1|
    \item \verb|j| $\leftarrow$ \verb|2 * 5|
    \item \verb|k| $\leftarrow$ \verb|2 / 5|
    \item \verb|r| $\leftarrow$ \verb|2.4|
    \item \verb|b| $\leftarrow$ \verb|0b1000| (liczba 8 zapisana dwójkowo)
    \item \verb|o| $\leftarrow$ \verb|0o11| (liczba 8 zapisana ósemkowo)
    \item \verb|h| $\leftarrow$ \verb|0h10| (liczba 16 zapisana szesnastkowo)
    \item \verb|test| $\leftarrow$ \verb|False|
    \item \verb|jezyk| $\leftarrow$ \verb|'Python'|
    \item \verb|ver| $\leftarrow$ \verb|'3.10'|
    \item \verb|i2| $\leftarrow$ \verb|3 ** 2|
    \item \verb|imie| $\leftarrow$ \verb|'Adam'|
    \item \verb|kody| $\leftarrow$ \verb|{'a': 0, 'b': 1, 'c': 3, 'd': 4}|
    \item \verb|odwiedzone| $\leftarrow$ \verb|{1, 2, 3, 4, 5}|
    \item \verb|pierwsze| $\leftarrow$ \verb|(2, 3, 5, 7)|
    \item \verb|tic_tac| $\leftarrow$ \verb|[[0, 0, 0], [0, 0, 0], [0, 0, 0]]|
\end{enumerate}

\textbf{\arabic{zadanie}.}\addtocounter{zadanie}{1} Zmień wartości utworzonych w poprzednim zadaniu zmiennych na następujące (zmiennej z lewej strony strzałki nadaj wartość z prawej strony):

\begin{multicols}{2}
\begin{enumerate}[label=\arabic*.]
    \item \verb|l| $\leftarrow$ \verb|2 * l|
    \item \verb|i| $\leftarrow$ \verb|-i|
    \item \verb|j| $\leftarrow$ \verb|j ** 2 + 5|
    \item \verb|k| $\leftarrow$ \verb|k / 5|
    \item \verb|r| $\leftarrow$ \verb|2.4 * r|
    \item \verb|test| $\leftarrow$ \verb|not test|
    \item \verb|jezyk| $\leftarrow$ \verb|'Nim'|
    \item \verb|ver| $\leftarrow$ \verb|'1.60'|
    \item \verb|o| $\leftarrow$ \verb|0o123|
    \item \verb|h| $\leftarrow$ \verb|0hfe|
    \item \verb|b| $\leftarrow$ \verb|0b111|
    \item \verb|imie| $\leftarrow$ \verb|'Ewa'|
\end{enumerate}
\end{multicols}

\textbf{\arabic{zadanie}.}\addtocounter{zadanie}{1} Utwórz zmienną \verb|t = 4|, a następnie za pomocą \emph{inkrementacji}:

\begin{enumerate}[label=\arabic*.]
    \item zwiększ jej wartość o 1
    \item pomnóż ją przez 4
    \item podziel (dzielenie całkowite) przez 2
    \item zmniejsz o 3
    \item podziel (dzielenie całkowite) przez 3
    \item podnieś do potęgi 3 (tak, to działa!)
    \item zredukuj modulo 5
\end{enumerate}


\section{Wyrażenia}

\textbf{\arabic{zadanie}.}\addtocounter{zadanie}{1} Wykonaj obliczenia --- staraj się robić to samodzielnie. Pamiętaj o kolejności działań! \textbf{Ustal typ wyniku}.

\begin{multicols}{3}
\begin{enumerate}[label=\arabic*.]
    \item \verb|2 * 4 - 5|
    \item \verb|2 * 4 - 5.|
    \item \verb|2 * 5 ** 2|
    \item \verb|2.0 * 5 ** 2|
    \item \verb|.5 * 2|
    \item \verb|0.5 * 4|
    \item \verb|1 / 2|
    \item \verb|1 // 2|
    \item \verb|3 // 4|
    \item \verb|4 // 2|
    \item \verb|4 // 2.|
    \item \verb|5 % 1|
    \item \verb|2 % 1.0|
    \item \verb|3 % 2.|
    \item \verb|2.7 % 1|
    \item \verb|2. % 1|
    \item \verb|5 % 3|
    \item \verb|17 % 3|

    \item \verb|17 // 3|
    \item \verb|15 // 4|
    \item \verb|16 // 4|
    \item \verb|16 / 4|
    \item \verb|3 / 4|
    \item \verb|3.0 / 4.|
    \item \verb|3. // 4|
    \item \verb|3 // 4.0|

    \item \verb|.1 * 10|

    \item \verb|2 ** 3|
    \item \verb|8 ** 0.5|
    \item \verb|abs(-2)|
    \item \verb|abs(-2) + abs(5)|
    \item \verb|pow(11, 5, 4)|
    \item \verb|pow(3, 7, 7)|
    \item \verb|divmod(11, 4)|
    \item \verb|divmod(11, 2.)|
    \item \verb|divmod(11.0, 2.)|

\end{enumerate}
\end{multicols}

\textbf{\arabic{zadanie}.}\addtocounter{zadanie}{1} Wykonaj obliczenia --- staraj się robić to samodzielnie. \textbf{Ustal typ wyniku}.

\begin{multicols}{2}
\begin{enumerate}[label=\arabic*.]
    \item \verb|4 * True - 3|
    \item \verb|2 + True|
    \item \verb|False * 1 / 7|
    \item \verb|True / 3 - 1 / 3|
    \item \verb|True ** 3 * False|
    \item \verb|2 ** False|
    \item \verb|2 * True / (True + False) |
    \item \verb|True ** 2|
    \item \verb|True ** 2.0|
    \item \verb|1.0 == True|
    \item \verb|1.5 == True|
    \item \verb|2 >= True|

    \item \verb|1 == True - True|
    \item \verb|(1 == True) - True|
    \item \verb|1 == True == True|
    \item \verb|1 == True == False|

\end{enumerate}
\end{multicols}

\textbf{\arabic{zadanie}.}\addtocounter{zadanie}{1} Ustal wartość logiczną wyrażeń.

\begin{multicols}{2}
\begin{enumerate}[label=\arabic*.]
    \item \verb|not False|
    \item \verb|False or True|
    \item \verb|not False == True|
    \item \verb|1 <= 1|
    \item \verb|2 >= 2|
    \item \verb|2 > 2|
    \item \verb|not 1 > 2|
    \item \verb|1 > 2 and True|
    \item \verb|1 <= 2 and True|
    \item \verb|1 > 2 or True|
    \item \verb|not 1 > 2 and True|
    \item \verb|-1 > 0 or -1 > -2|
    \item \verb|-1 < 0 or -1 > 1|
    \item \verb|-1 < 0 and 1 > 1|
    \item \verb|-1 < 0 or 1 > 1|
    \item \verb|1 == True|
    \item \verb|False == 0|
    \item \verb|1 <= 0 <= 3|
    \item \verb|3 <= 3 <= 3|
    \item \verb|1 <= 1 <= 2|
    \item \verb|1 < 2 <= 2|
    \item \verb|1 < 2 < 2|

\end{enumerate}
\end{multicols}


\textbf{\arabic{zadanie}.}\addtocounter{zadanie}{1} Nieco trudniejsze przykłady.

\important{Warto pamiętać, że \texttt{or} i \texttt{and} tworzą \emph{wyrażenia}. Zwracają one jako wartość ten element, który rozstrzyga o wartości logicznej wyrażenia.}

Na przykład \verb|0 or 3 or False| zwróci wartość \verb|3|, bo \verb|3| powoduje, że wartością wyrażenia jest \verb|True|.

Podobnie, \verb|1 and [] and True| zwróci wartość \verb|[]|, bo pusta lista \verb|[]| jest wartościowana jako \verb|False| i to przesądza o wartości logicznej wyrażenia.

\begin{multicols}{2}
\begin{enumerate}[label=\arabic*.]
    \item \verb|True or ''|
    \item \verb|False or 'a'|
    \item \verb|2 or True|
    \item \verb|True and True and ''|
    \item \verb|0 or False or 'kot'|
    \item \verb|2 and 3|
    \item \verb|True and 2|
    \item \verb|2 and False or 4|
    \item \verb|1 and []|
    \item \verb|[] or 2|

\end{enumerate}

\end{multicols}

\section{Rozpakowanie}

\textbf{\arabic{zadanie}.}\addtocounter{zadanie}{1} Rozpakowanie to inaczej \emph{destrukturyzacja}. Być może ten termin jest częściej używany w świecie ,,profesjonalistów''.

\paragraph{Przykłady}

\begin{enumerate}[]
    \item \verb|a, b, c = [1, 2, 3]| $\rightarrow$ \verb|a = 1, b = 2, c = 3|
    \item \verb|a, b, c = 'kuc'| $\rightarrow$ wówczas \verb|a = 'k', b = 'u', c = 'c'|
    \item \verb|a, b = [0, 10]| $\rightarrow$  \verb|a = 0, b = 10|
    \item \verb|a, b, c = [0, 10, 20]| $\rightarrow$  \verb|a = 0, b = 10, c = 20|
\end{enumerate}

Naśladując powyższe, uzupełnij tabelę poniżej.

\begin{tabular}{p{.4\textwidth}|p{.5\textwidth}}
\textsf{rozpakowanie} & \textsf{wynik} \\\hline
\verb|a, b = [1, 2]| & \dots \\
\verb|a, b, c = [1, 2, 3]| & \dots \\
\verb|a, b, c, d = [1, 2, 3, 4]| & \dots \\
\verb|a, b, c, d = 'kura'| & \dots \\
\verb|a, b, c, d = 'kot'| & \dots \\
\end{tabular}

\textbf{\arabic{zadanie}.}\addtocounter{zadanie}{1} Ciąg dalszy.

\important{Jeśli liczba zmiennych po lewej stronie znaku \texttt{=} nie jest równa liczbie elementów obiektu po prawej, \emph{musimy} gdzieś użyć gwiazdki \texttt{*}.}

\paragraph{Przykłady}

\begin{enumerate}[]
    \item \verb|a, *b = [1, 2, 3]| -- wówczas \verb|a = 1, b = [2, 3]|
    \item \verb|*a, b = [1, 2, 3]| -- wówczas \verb|a = [1, 2], b = 3|
    \item \verb|a, b, *c = 'kurczak'| -- wówczas \verb|a = 'k', b = 'u', c = ['r', 'c', 'z', 'a', 'k']|
    \item \verb|a, b, *c = [1, 2, 3]| -- wówczas \verb|a = 1, b = 2, c = [3]|
    \item \verb|a, *b, c = [0, 1, 2, 3, 4]| -- \verb|a = 0, b = [1, 2, 3], c = 4|
    \item \verb|*a, b, c = [0, 1, 2, 3, 4]| -- \verb|a = [0, 1, 2], b = 3, c = 4|
\end{enumerate}

Naśladując powyższe, uzupełnij tabelę poniżej.

\begin{tabular}{p{.4\textwidth}|p{.5\textwidth}}
\textsf{rozpakowanie} & \textsf{wynik} \\\hline
\verb|a, *b = [1, 2, 3, 4]| & \dots \\
\verb|*a, b = [1, 2, 3, 4]| & \dots \\
\verb|a, *b, c = [1, 2, 3]| & \dots \\
\verb|a, *b = 'kucyk'| & \dots \\
\verb|*a, b, c = 'kucyk'| & \dots \\
\verb|a, *b, c = 'kucyk'| & \dots \\
\verb|a, *b = 'kucyk'| & \dots \\
\verb|*a, b = 'kucyk'| & \dots \\
\end{tabular}

\textbf{\arabic{zadanie}.}\addtocounter{zadanie}{1} Po lewej stronie znaku \verb|=| napisz w odpowiedni sposób zmienne, by otrzymać podane wyniki.

\begin{tabular}{p{.4\textwidth}|p{.5\textwidth}}
\textsf{rozpakowanie} & \textsf{wynik} \\\hline
\verb|... = [1, 2, 3, 4]| & \verb|a = [1, 2], b = 3, c = 4| \\
\verb|... = 'maska'| & \verb|a = ['m', 'a', 's', 'k'], b = 'a'| \\
\verb|... = 'maska'| & \verb|a = ['m', 'a', 's'], b = 'k', c = 'a'| \\
\verb|... = 'maska'| & \verb|a = 'm', b = 'a', c = ['s', 'k', 'a']| \\
\verb|... = 'maska'| & \verb|a = 'm', b = ['a', 's', 'k'], c = 'a'| \\
\verb|... = [1, 2, [3, 4]]| & \verb|a = [1, 2], b = [3, 4]| \\
\end{tabular}


\section{Wyrażenie warunkowe}

\textbf{\arabic{zadanie}.}\addtocounter{zadanie}{1} Wyznacz wartość wyrażeń warunkowych w poniższej tabeli.

\begin{tabular}{l|l|c}
\textsf{zmienna} & \textsf{wyrażenie warunkowe} & \textsf{wartość} \\\hline
\verb|s = 0b10| & \verb|3 if s == 2 else 0| & 3\\
\verb|s = 0o14| & \verb|s if s >= 12 else 0| &\\
\verb|s = 1| & \verb|1 if s else 0| &\\
\verb|s = 'a_0'| & \verb|1 if s else 0|&\\
\verb|s = 'a_0'| & \verb|1 if len(s) <= 3 else 0|&\\
\verb|s = 'a_0'| & \verb|'x' if s[0] == 'a' else 'y'|&\\
\verb|s = 'a_0'| & \verb|'x' if s[-1] != '2' else 'y'|&\\
\verb|s = 'a_0'| & \verb|'x' if s[-1] == '0' else 'y'|&\\
\verb|s = 'a_0'| & \verb|'x' if s[:0] else 'y'|&\\
\verb|s = ''| & \verb|3 if s else 9|&\\
\verb|s = []| & \verb|0 if s else 1|&\\
\verb|s = {}| & \verb|0 if s else 1|&\\
\end{tabular}
\newline


\textbf{\arabic{zadanie}.}\addtocounter{zadanie}{1} Niech \verb|z = 'alf'|. Oblicz samodzielnie wartość wyrażeń.

\begin{multicols}{2}
\begin{enumerate}[label=\arabic*.]
    \item \verb|len(z) >= 3|
    \item \verb|len(z) < 4|
    \item \verb|len(z) >= 1|
    \item \verb|'l' in z|
    \item \verb|'fl' in z|
    \item \verb|'af' in z|
    \item \verb|'alf' in z|
    \item \verb|'lf' in z|
    \item \verb|1 if len(z) == 3 else -1|
    \item \verb|2 if len(z) <= 3 else 0|
    \item \verb|2 if len(z) % 2 == 1 else 0|

\end{enumerate}
\end{multicols}


\section{Listy}

\textbf{\arabic{zadanie}.}\addtocounter{zadanie}{1} Utwórz szybko listę złożoną:
\begin{enumerate}[label=\arabic*.]
    \item ze stu liczb równych \verb|0|
    \item z tysiąca liczb równych \verb|1|
    \item z pięciuset liter \verb|'a'|
\end{enumerate}


\textbf{\arabic{zadanie}.}\addtocounter{zadanie}{1} Wykonaj poniższe polecenia.
\begin{enumerate}[label=\arabic*.]
    \item Napisz wyrażenie, które tworzy listę złożoną z liczb \verb|1, 2, 3|.
    %~ [1, 2, 3]
    \item Napisz inne wyrażenie, które tworzy taką samą listę.
    %~ list(range(1, 4))
    %~ [j for j in range(1, 4)]
    \item Przypisz tę listę zmiennej \verb|l|.
    \item Napisz pięć różnych wyrażeń, które z pomocą listy \verb|l| tworzą listę \verb|[1, 2, 3, 4]|.
    %~ l.append(4) // l.extend([4]) // l += [4] // l = [*l, 4] // l[-1:] = [3, 4]
    \item Napisz dwa wyrażenia, które z listy \verb|l| usuwają ostatni element.
    %~ l.pop(), del l[-1]
\end{enumerate}

\textbf{\arabic{zadanie}.}\addtocounter{zadanie}{1} Wykonaj sekwencję poleceń, używając odpowiednich konstrukcji Pythona, funkcji i metod.

\begin{enumerate}[label=\arabic*.]
    \item Utwórz listę \verb|l=[5, 5, 5]|.
    \item Na końcu listy \verb|l| dodaj kolejno elementy \verb|7, 8, 9|.
    \item Między elementy \verb|5| i \verb|7| wstaw \verb|6|.
    \item Zerowy element listy \verb|l| zamień na \verb|3|, pierwszy na \verb|4|
    \item Usuń ostatni element listy \verb|l|.
    \item Rozszerz listę \verb|l| o listę \verb|[10, 11, 12]|.
    \item Usuń trzeci (tj. mający indeks 2) element listy \verb|l|.
\end{enumerate}


\textbf{\arabic{zadanie}.}\addtocounter{zadanie}{1} Dana jest lista \verb|l = [1, 2, 3, 4, ..., 20]|. Używając notacji ,,slice'' \verb|[i:j:k]| skopiuj z niej poniższe listy:

\begin{multicols}{2}
\begin{enumerate}[label=\arabic*.]
    \item \verb|[1, 2, 3, ..., 10]|
    \item \verb|[10, 11, 12, ..., 20]|
    \item \verb|[2, 4, 6, 8, 10]|
    \item \verb|[10, 12, 14, 16, 18, 20]|
    \item \verb|[1, 3, 5, 7, 9]|
    \item \verb|[11, 13, 15, 17, 19]|
    \item \verb|[1, 3, 5, ..., 19]|
    \item \verb|[2, 4, 6, ..., 20]|
    \item \verb|[1, 4, 7, ..., 19]|
    \item \verb|[2, 5, 8, ..., 20]|
\end{enumerate}

\end{multicols}


\textbf{\arabic{zadanie}.}\addtocounter{zadanie}{1} Używając ,,slice'' zmodyfikuj listę \verb|l| z poprzedniego zadania, jak poniżej.

\begin{enumerate}[label=\arabic*.]
    \item \verb|[0, 0, ..., 0, 10, 11, ..., 20]|
    \item \verb|[1, 2, ..., 9, 10, 10, ..., 10]|
    \item \verb|[1, 2, 19, 20]|
    \item \verb|[1, 2, 3, 18, 19, 20]|
    \item \verb|[1, 2, 3]|
    \item \verb|[1, 2, 19, 20]|
    \item \verb|[1, 2, ..., 7, 8, 8, 8, 8, 12, 13, ..., 20]|
    \item \verb|[1, 2, ..., 9, 10, 10, ..., 10]|
    \item \verb|[7, 2, 7, 4, 7, 6, 7, 8, 7, 10, 7, 12, 7, 14, 7, 16, 7, 18, 7, 20]|

\end{enumerate}

\textbf{\arabic{zadanie}.}\addtocounter{zadanie}{1} Dana jest lista \verb|l| taka, jak w poprzednim zadaniu. Napisz wyrażenia, które korzystają z \verb|l| i dają w wyniku poniższe listy.

\begin{multicols}{2}
\begin{enumerate}[label=\arabic*.]
    \item \verb |[1, 3, 5, 7, 9]|
    \item \verb |[1, 2, 3, 4, 5]|
    \item \verb |[2, 4, 6, 8, 10]|
    \item \verb |[1, 4, 7, 10]|
    \item \verb |[2, 6, 10]|
    \item \verb |[2, 5, 8]|
    \item \verb |[3, 6, 9]|
    \item \verb |[2, 6]| (dwoma sposobami)
    \item \verb |[5, 8]| (dwoma sposobami)
    \item \verb |[3, 9]| (dwoma sposobami)
    \item \verb |[3, 4, 5, 6]|
    \item \verb |[15, 16, 17, 18, 19, 20]|
\end{enumerate}
\end{multicols}


\section{List comprehension}
List comprehension to bardzo wygodna notacja, którą można stosować nie tylko w odniesieniu do list.

\important{Ogólna składnia \emph{list comprehension} jest następująca:

\vspace{.5em}
\centerline{\texttt{[ ... for x in} \emph{ob\_iterowalny} \texttt{if war\_logiczny]}}
\vspace{.5em}
Powyższy wariant tworzy listę -- użycie nawiasu klamrowego utworzy zbiór lub słownik, w zależności od szczegółów konstrukcji.

\vspace{.5em}

\texttt{...} należy rozumieć jako realizację pewnej operacji.

\vspace{.5em}

\textbf{Uwaga:} warunek logiczny nie jest obowiązkowy, wszystko zależy od konkretnej sytuacji.
}

\paragraph{Przykład} Niech \emph{ob\_iterowalny}\verb| = 'Python'|.
\begin{enumerate}[]
    \item \verb|[ord(x) for x in 'Python']| $\rightarrow$ \verb|[80, 121, 116, 104, 111, 110]|
    \item \verb|[i for i, c in enumerate('Python')]| $\rightarrow$ \verb|[0, 1, 2, 3, 4, 5]|
    \item \verb|{i: c for i, c in enumerate('Python')}| $\rightarrow$\newline
    \verb|{0: 'P', 1: 'y', 2: 't', 3: 'h', 4: 'o', 5: 'n'}|
    \item \verb|{c for c in 'Python' if c != 'o'}| $\rightarrow$ \verb|{'P', 'y', 't', 'h', 'n'}| (to jest zbiór, zatem w rzeczywistości kolejność liter może być różna)
    \item \verb|[c for i, c in enumerate('Python') if i % 2]| $\rightarrow$ \verb|['y', 'h', 'n']|
\end{enumerate}

\textbf{\arabic{zadanie}.}\addtocounter{zadanie}{1} Dane są zmienne
\begin{enumerate}[]
    \item \verb|a = 'kosmodrom'|
    \item \verb|b = 'alfabetyzacja'|
    \item \verb|c = 'luminescencja'|
    \item \verb|x = [1, 2, 3, 4, 5, 6, 7, 8, 9, 10]|
\end{enumerate}

Korzystając z tych zmiennych napisz list comprehension, którego wynikiem jest:
\begin{enumerate}[label=\arabic*.]
    \item \verb|['k', 'o', 's', 'm', 'o', 'd', 'r', 'o', 'm']|
    \item \verb|['a', 'l', 'f', 'a', 'b', 'e', 't', 'y', 'z', 'a', 'c', 'j', 'a']|
    \item \verb|['l', 'u', 'm', 'i', 'n', 'e', 's', 'c', 'e', 'n', 'c', 'j', 'a']|
    \item \verb|['l', 'u', 'm', 'i', 'n', 'e', 's', 'e', 'n', 'j', 'a']|
    \item \verb|['l', 'm', 'n', 's', 'c', 'n', 'c', 'j']|
    \item \verb|[2, 4, 6, 8, 10, 12, 14, 16, 18, 20]|
    \item \verb|[1, 4, 9, 16, 25, 36, 49, 64, 81, 100]|
    \item \verb|[1, 3, 5, 7, 9]|
    \item \verb|[1, 2, 3, 4, 5]|
    \item \verb|[2, 4, 6, 8, 10]|
    \item \verb|[15, 16, 17, 18, 19, 20]|
    \item \verb|['a', 'a', 'a', 'a']|
    \item \verb|['a', 'a', 'b', 'a', 'c', 'a']|
    \item \verb|['o', 'o', 'o']|
    \item \verb|[97, 108, 102, 97, 98, 101, 116, 121, 122, 97, 99, 106, 97]|
\end{enumerate}

\textbf{\arabic{zadanie}.}\addtocounter{zadanie}{1} Dana jest lista
\begin{verbatim}
l = ['koc', 'noc', 'nos', 'nic',
    'kloc', 'kos', 'moc', 'kic',
    'kłos', 'nas', 'kasa', 'klasa',
    'masa', 'las', 'los', 'lis']
\end{verbatim}
Napisz list comprehension, którego wynikiem jest lista:
\begin{enumerate}[label=\arabic*.]
    \item \verb|['koc', 'kloc', 'kos', kic', 'kłos', 'kasa', 'klasa']|
    \item \verb|['koc', 'noc', 'nic', 'kloc', 'moc', 'kic']|
    \item \verb|['koc', 'noc', 'kloc', 'moc']|
    \item \verb|['nas', 'kasa', 'klasa', 'masa', 'las']|
    \item \verb|['nos', 'kos', 'kłos', 'los']|
    \item \verb|['klasa', 'las']|
    \item \verb|['kloc', 'klasa', 'las', 'los', 'lis']|
    \item \verb|['koc', 'noc', 'nos', 'nic', 'kos', 'moc', 'kic','nas', 'las', 'los', 'lis']|
\end{enumerate}

\section{Teksty}
\textbf{\arabic{zadanie}.}\addtocounter{zadanie}{1} Za pomocą operatora \verb|*| utwórz tekst złożony z:
\begin{enumerate}[label=\arabic*.]
    \item ze stu liter \verb|'a'|
    \item czterdziestu myślników \verb|'-'|
    \item trzydziestu gwiazdek \verb|'*'|
\end{enumerate}


\textbf{\arabic{zadanie}.}\addtocounter{zadanie}{1} Za pomocą operatora \verb|+| utwórz z tekstów \verb|'Krowa'|, \verb|'daje'|, \verb|'mleko'| teksty
\begin{enumerate}[label=\arabic*.]
    \item \verb|'Krowa daje mleko'|
    \item \verb|'Krowa mleko daje'|
    \item \verb|'mleko daje Krowa'|
\end{enumerate}


\textbf{\arabic{zadanie}.}\addtocounter{zadanie}{1} Z trzech tekstów \verb|'Trzeba'|, \verb|'więcej'|, \verb|'pracować'| utwórz teksty
\begin{enumerate}[label=\arabic*.]
    \item \verb|'Trzeba więcej pracować'|
    \item \verb|'więcej Trzeba pracować'|
    \item \verb|'pracować Trzeba więcej'|
    \item \verb|'więcej pracować Trzeba'|
    \item \verb|'Trzeba pracować więcej'|
    \item \verb|'pracować więcej Trzeba'|
\end{enumerate}

\section{Metody obiektu \texttt{str}}

\textbf{\arabic{zadanie}.}\addtocounter{zadanie}{1} Z trzech tekstów \verb|'Trzeba'|, \verb|'więcej'|, \verb|'pracować'| utwórz poniższe teksty. Wykorzystaj metodę \verb|capitalize()|.
\begin{enumerate}[label=\arabic*.]
    \item \verb|'Trzeba Więcej Pracować'|
    \item \verb|'Więcej Trzeba Pracować'|
    \item \verb|'Pracować Trzeba Więcej'|
    \item \verb|'więcej Pracować Trzeba'|
    \item \verb|'Trzeba Pracować Więcej'|
    \item \verb|'Pracować Więcej Trzeba'|
\end{enumerate}


\textbf{\arabic{zadanie}.}\addtocounter{zadanie}{1} Utwórz zmienną \verb|motto| i nadaj jej wartość \verb|'nie lubię myć nóg'|. Użyj odpowiedniej metody, by uzyskać tekst:

\begin{enumerate}[label=\arabic*.]
    \item 'NIE LUBIĘ MYĆ NÓG'
    \item 'Nie lubię myć nóg'
    \item 'Nie Lubię Myć Nóg'
\end{enumerate}

\textbf{\arabic{zadanie}.}\addtocounter{zadanie}{1} Korzystając z tekstu \verb|'nie lubię myć nóg'| i wskazanych metod wykonaj poniższe polecenia:

\begin{enumerate}[label=\arabic*.]
    \item \verb|split()| -- utwórz zmienną \verb|l| i nadaj jej wartość \verb|['nie', 'lubię', 'myć', 'nóg']|
    \item \verb|join()| -- zamień element \verb|'nie'| na \verb|'Nie'|, a następnie zmiennej \verb|ll| nadaj wartość \verb|'Nie-lubię-myć-nóg'|
    \item \verb|count()| -- podaj, ile razy występuje w tym tekście \verb|'i'|
    \item \verb|rindex()| -- jaki jest indeks ostatniej litery \verb|'i'| w tym tekście
    \item \verb|find()| -- podaj indeks, pod jakim w danym tekście występuje tekst \verb|'my'|
    \item \verb|split()| -- teraz nadaj zmiennej \verb|l| wartość \verb|['nie', 'lubię', 'myć', 'nóg']|
    \item \verb|join()| -- zmiennej \verb|l| nadaj wartość \verb|'Nie ** lubię ** myć ** nóg'|
    \item \verb|replace()| -- zmiennej \verb|l| nadaj teraz wartość \verb|'Nie -- lubię -- myć -- nóg'|
    \item \verb|startswith()| -- sprawdź, czy tekst \verb|l| zaczyna się od \verb|'Nie'|
    \item \verb|islower()| -- sprawdź, czy tekst \verb|l| składa się wyłącznie z małych liter
    \item \verb|startswith()| -- sprawdź, czy tekst \verb|l| rozpoczyna się od \verb|'nie'|
    \item \verb|endswith()| -- sprawdź, czy tekst \verb|l| kończy się na \verb|'róg'|

\end{enumerate}

\textbf{\arabic{zadanie}.}\addtocounter{zadanie}{1} Dane są zmienne
\begin{enumerate}[]
    \item \verb|a = 'elektroluminescencja'|,
    \item \verb|b = 'alfabetyzaja'|,
    \item \verb|c = 'kosmodrom'|.
\end{enumerate}

Napisz wyrażenie, którego wynikiem jest tekst
\begin{enumerate}[label=\arabic*.]
    \item \verb|'alfabetyzacjakosmodrom'|
    \item \verb|'alf'|
    \item \verb|'kos'|
    \item \verb|'elektro'|
    \item \verb|'alfabet'|
    \item \verb|'luminescencja'|
    \item \verb|'m'| - ostatni znak tekstu c
    \item \verb|'om'|
    \item \verb|'k'| - pierwszy znak tekstu c
    \item \verb|'osm'|
    \item \verb|'trol'|
    \item \verb|'mordomsok'|
    \item \verb|lista ['k', 'o', 's', 'm', 'o', 'd', 'r', 'o', 'm']|
    \item \verb|'ajcazytebafla'|

    \item \verb|'***kosmodrom***'|
    \item \verb|'Kosmodrom'|
    \item \verb|'KOSMODROM'|


\end{enumerate}

\section{Formatowanie tekstów}
Funkcja \lstinline|print()| wyświetla dane na ekranie monitora.

Python udostępnia kilka sposobów formatowania informacji, która ma być wyświetlona. Pokażę tu jedynie kilka przykładów, szczegółowy opis znajdziesz w skrypcie lub internecie.

Korzystanie z metody \lstinline|format()| wymaga podania tekstu, który ją wywoła; jednocześnie sam tekst posłuży jako szablon. W obrębie szablonu szczególne znaczenie mają nawiay klamrowe \verb|{}|, do środka których Python wstawi argumenty podane metodzie, zinterpretuje je i utworzy wynikowy tekst.

\paragraph{Przykłady} Niech \verb|a = 421.07832|
\begin{enumerate}[label=\alph*)]
    \item \verb|'{}'.format(a)| $\rightarrow$ \verb|421.07832|
    \item \verb|'{:.2}'.format(a)| $\rightarrow$ \verb|4.2e+02|
    \item \verb|'{:.3f}'.format(a)| $\rightarrow$ \verb|421.078|
    \item \verb|'{:2.2}'.format(a)| $\rightarrow$ \verb|4.2e+02|
    \item \verb|'{:10.2f}'.format(a)| $\rightarrow$ \texttt{\textvisiblespace\textvisiblespace\textvisiblespace\textvisiblespace421.08}
    \item \verb|'{:+^10.2f}'.format(a)| $\rightarrow$ \verb|++421.08++|
    \item \verb|'{:<10.2f}'.format(a)| $\rightarrow$ \texttt{421.08\textvisiblespace\textvisiblespace\textvisiblespace\textvisiblespace}
    \item \verb|'{:.>15.2f}'.format(a)| $\rightarrow$ \verb|.........421.08|
    \item \verb|'{0:.>15.2f}'.format(a)| $\rightarrow$ \verb|.........421.08|
    \item \verb|'- {} -'.format(a)| $\rightarrow$ \verb|- 421.0783 -|
    \item \verb|'{:.2f} - {}'.format(a, 2)| $\rightarrow$ \verb|'421.08 - 2'|
\end{enumerate}
\begin{enumerate}[label=\alph*)]
    \item Najprostsze wykorzystanie --- zamiast \verb|{}| Python wstawi zmienną \verb|a| i utworzy odpowiedni tekst.
    \item Po dwukropku wewnątrz nawiasów \verb|{}| opisujemy w jaki sposób ma zostać wyświetlona liczba. Liczba \emph{po kropce} określa ile cyfr jest przeznaczonych na zapis \emph{całej liczby}. Dla zapisu liczby \verb|421.07832| za pomocą \emph{dwóch cyfr}, Python musi użyć notacji wykładniczej. W naszym przypadku wyświetlona liczba i tak będzie przybliżeniem prawdziwej wartości zmiennej \verb|a|.
    \item Jeżeli po kropce użyto dodatkowo specyfikatora \verb|f| (od \emph{float}, co oznacza liczbę rzeczywistą), to liczba po kropce określa teraz ile miejsc po przecinku ma zostać wyświetlonych. W naszym przypadku sprowadza się to do zaokrąglenia liczby do trzech miejsc dziesiętnych.
    \item Liczba \emph{przed kropką} oznacza ile miejsc przeznaczonych jest na \emph{wyświetlenie} liczby. Jeżeli liczba ta jest mniejsza niż wynika to z realiów, jest ignorowana --- tak jest w naszym przypadku.
    \item Tutaj mamy dziesięć miejsc na wyświetlenie liczby, z czego dwa na cyfry dziesiętne (\verb|f| za dwójką po kropce). W naszym przypadku  potrzebnych jest sześć miejsc, a więc cztery pozostają niewykorzystane --- Python domyślnie uzupełni je spacjami.
    \item Różnica między poprzednim przykładem sprowadza się do znaków \verb|+^|w szablonie. Należy to rozumieć tak: \emph{niewykorzystane miejsca} wypełnić znakami \verb|+|, a liczbę \emph{wyśrodkować} w obrębie dziesięciu miejsc.
    \item Jeżeli nie podano ,,wypełniacza'', to domyślnie jest on spacją. Należy zatem liczbę wyrównać do lewej strony obszaru dziesięciu miejsc.
    \item Piętnaście miejsc na wyświetlenie, kropka \verb|.| jako wypełniacz, wyrównanie do prawej.
    \item Argumenty funkcji \verb|format()| zamieniają nawiasy \verb|{}| w kolejności ich występowania. W razie potrzeby możemy tę kolejność zmienić, wskazująć miejsce w którym ma zostać podstawiony konkretny argument. W naszym przypadku nie ma to znaczenia.
    \item Szablon służy wyłącznie jako ozdobnik.
    \item Pierwszy parametr zastępuje pierwszy nawias \verb|{}|, drugi --- drugi.
\end{enumerate}
Czas na zadania.

\textbf{\arabic{zadanie}.}\addtocounter{zadanie}{1} Dane są zmienne:
\begin{enumerate}[]
    \item \verb|e = 3.21|
    \item \verb|f = 16/7|
    \item \verb|j = 7|
    \item \verb|str = 'szpak'|
    \item \verb|str_2 = 'kos'|
\end{enumerate}

Wypisz wartości wskazanych zmiennych, korzystając z metody \verb|format()|. Użyj podanego tekstu jako szablonu.

\begin{enumerate}[label=\arabic*.]
    \item \verb|e, f, j| -- \verb|'{} {} {}'|
    \item \verb|e, f, j| -- \verb|'e={} f={} j={}'|

    \item \verb|e, f, j| -- \verb|'{0}, {1}, {2}'|
    \item \verb|e, f, j| -- \verb|'e={0}, f={1}, j={2}'|

    \item \verb|e, f, j| -- \verb|'f={1}, j={2}, e={0}'|
    \item \verb|e, f, j| -- \verb|'{0}-{0}-{2}'|
    \item \verb|e, f, j| -- \verb|'{2} / {1} / {0}'|
    \item \verb|e, f, j| -- \verb|'{1} : {0} : {2}'|

    \item \verb|e, f, j| -- \verb|'{:_^8}-{}-{:*^7}'|
    \item \verb|e, f, j| -- \verb-'{0:_<8} | {1} | {2:*^9}'-

    \item \verb|e, f, j| -- \verb|'e={: >8}, f={:<12.3} {:*^7}'|
    \item \verb|e, f, j| -- \verb-'e={0:_<8} | f={1:+^10.4} | {2:*^9}'-

    \item \verb|f| -- \verb|'f={:+^10.1}|
    \item \verb|f| -- \verb|'f={:+<10.2}|
    \item \verb|f| -- \verb|'f={:+<10.3}|
    \item \verb|f| -- \verb|'f={:+>10.4}|
    \item \verb|f| -- \verb|'f={:+>10.5}|
    \item \verb|f| -- \verb|'f={:+>10.6}|

    \item \verb|str, str_2| -- \verb|'{} i {}'|
    \item \verb|str, str_2| -- \verb|'{0} i {1}'|
    \item \verb|str, str_2| -- \verb|'{1} i {0}'|
    \item \verb|str, str_2| -- \verb|'{1} i {2}'|
    \item \verb|str, str_2| -- \verb|'{} i {}'|

\end{enumerate}

\textbf{\arabic{zadanie}.}\addtocounter{zadanie}{1} Korzystając ze zmiennych z poprzedniego zadania zbuduj szablon dla funkcji \verb|format()| i przekaż jej odpowiednie parametry, by uzyskać wskazany wynik.

\begin{enumerate}[label=\arabic*.]
    \item \verb|'3.21 + 2.2857142857142856 = 5.4957142857142856'|
    \item \verb|'3.21 + 2.29 = 5.50'|
    \item \verb|'szpak i szpak'|
    \item \verb|'szpak i SZPAK'|
    \item \verb|'7, 3, 1'|

\end{enumerate}

\textbf{\arabic{zadanie}.}\addtocounter{zadanie}{1} Wykonaj poprzednie zadanie z wykorzystaniem f-stringów zamiast \verb|format()|.

\section{Konwersja na różne układy}

\textbf{\arabic{zadanie}.}\addtocounter{zadanie}{1} Wykorzystaj funkcję \verb|int| do zamiany liczby z danego układu na dziesiętny (pamiętaj, by liczbę podać jako string).

\begin{multicols}{3}
\begin{enumerate}[label=\arabic*.]
    \item $3_{3}\rightarrow \dots$
    \item $11_{3}\rightarrow \dots$
    \item $12_{3}\rightarrow \dots$
    \item $22_{3}\rightarrow \dots$

    \item $10_{2}\rightarrow \dots$
    \item $11_{2}\rightarrow \dots$
    \item $111_{2}\rightarrow \dots$
    \item $101_{2}\rightarrow \dots$
    \item $1111_{2}\rightarrow \dots$
    \item $1000_{2}\rightarrow \dots$
    \item $10000_{2}\rightarrow \dots$

    \item $10_{8}\rightarrow \dots$
    \item $11_{8}\rightarrow \dots$
    \item $17_{8}\rightarrow \dots$
    \item $77_{8}\rightarrow \dots$
    \item $22_{8}\rightarrow \dots$

    \item $\mathrm{A_{16}}\rightarrow \dots$
    \item $\mathrm{B_{16}}\rightarrow \dots$
    \item $\mathrm{F_{16}}\rightarrow \dots$
    \item $\mathrm{FF_{16}}\rightarrow \dots$

\end{enumerate}
\end{multicols}

\textbf{\arabic{zadanie}.}\addtocounter{zadanie}{1} Sprawdź, czy prawdziwe są równości. Tam gdzie można, wpisz liczbę bezpośrednio w podanym systemie liczbowym.

\begin{multicols}{3}
\begin{enumerate}[label=\arabic*.]
    \item $1010_{2} = 10$
    \item $11_{2} = 10_{3}$
    \item $11_{16} = 21_{8}$
    \item $\mathrm{A_{16}} = 10$
    \item $\mathrm{A_{16}} = 12_{8}$

    \item $11_{4} = 4$
    \item $11_{4} = 5$
    \item $4_{5} = 4$
    \item $4_{5} = 100_{2}$

    \item $\mathrm{FA_{16}} = 235$
    \item $\mathrm{AA_{16}} = 170$
    \item $\mathrm{E_{16}} = 86_{8}$

\end{enumerate}
\end{multicols}

\textbf{\arabic{zadanie}.}\addtocounter{zadanie}{1} Podaj reprezentacje danych liczb w układach szesnastkowym, ósemkowym i dwójkowym:

\begin{tabular}{c | c | c | c}
\textsf{dec} & \textsf{hex} & \textsf{oct} & \textsf{bin}\\\hline
12  & C & 14 & 1100 \\\hline
    & F &    &  \\\hline
51  &   &    &  \\\hline
    &   & 111 & \\\hline
    & EE &  &   \\\hline
    &   &   & 111111 \\\hline
256 &   &   &   \\\hline
    &   & 1234 &    \\\hline
    & A1A &     &   \\

\end{tabular}


\section{Prosty import}

\textbf{\arabic{zadanie}.}\addtocounter{zadanie}{1} Z modułu \verb|math| zaimportuj następujące funkcje: \verb|sqrt, floor, ceil, log, log2, log10| oraz stałą \verb|pi| i wykorzystaj je, by obliczyć następujące wyrażenia.

\begin{multicols}{3}
    \begin{enumerate}[label=\arabic*.]
        \item $\sqrt{9}$
        \item $\lceil{\sqrt{8}}\rceil$
        \item $\log_{2} 8$
        \item $\lceil{-2.7}\rceil$
        \item $\log_{10} 0{,}1$
        \item $\lfloor{12/3}\rfloor$
        \item $\lceil{12/3}$
        \item $\log_{2} 4$
        \item $\sqrt{16}$
        \item $\log_{2} 16$
        \item $\lfloor{1}\rfloor$
        \item $\lfloor{-2.7}\rfloor$
        \item $\lfloor{12/3}\rfloor$
        \item $\log_{10} 100$
        \item $\log_{2} 16$
        \item $\lceil{12/3}\rceil$
        \item $\lceil{\pi}\rceil$
        \item $\log_{10} 10 $
        \item $\log_{2} 128$
        \item $\lfloor{\pi}\rfloor$
        \item $\log_{10} 1$
    \end{enumerate}
\end{multicols}


%~ \subsubsection{Instrukcja warunkowa}


%~ Wzorując się na tym, uprość warunki w następujących instrukcjach:

%~ \begin{enumerate}[label=\arabic*.]
    %~ \item \verb|if x == True:|
    %~ \item \verb|if x != 0:|
    %~ \item \verb|if x > 0 and x > 1:|
    %~ \item \verb|if x < 0 or x >0:|
    %~ \item \verb|if len(x) > 0:|
    %~ \item \verb|if len(x) != 0:|
    %~ \item \verb|if x != '':|
    %~ \item \verb|if x != []:|
    %~ \item \verb|if not x == '':|
    %~ \item \verb|if len(x) > 0 and y in x:|

%~ \end{enumerate}


\section{Pętle \texttt{for} i \texttt{while}}
\newcounter{whileloop}\setcounter{whileloop}{\arabic{zadanie}}
\textbf{\arabic{zadanie}.}\addtocounter{zadanie}{1} Napisz pętlę \verb|for|, która z odpowiednio dobranym obiektem \verb|range()| wypisuje:

\begin{enumerate}[label=\arabic*.]
    \item liczby $1, 2, \dots 20$ w podanej kolejności
    \item liczby $-5, -4, \dots 10$ w podanej kolejności
    \item liczby $-15, -14, \dots 8$ w podanej kolejności
    \item liczby $0, 2, 4, \dots 16$ w podanej kolejności
    \item liczby $0, -2, -4, \dots -16$ w podanej kolejności
    \item liczby $-3, 0, 3, \dots 21$ w podanej kolejności
    \item liczby $-3, 1, 5, \dots 25$ w podanej kolejności
    \item liczby $-7, -5, -3, \dots 19$ w podanej kolejności
    \item liczby $7, 12, 17, \dots 47$ w podanej kolejności
    \item liczby $20, 17, 14, \dots -11$ w podanej kolejności
\end{enumerate}

\textbf{\arabic{zadanie}.}\addtocounter{zadanie}{1} Za pomocą pętli \verb|for| wypisz:
\begin{enumerate}[label=\arabic*.]
    \item liczby nieparzyste z zakresu 1-31 włącznie
    \item kwadraty liczb od 0 do 10 włącznie
    \item pierwiastki liczb od 4 do 16 włącznie
    \item reszty modulo 13 z piętnastu kolejnych liczb postaci $n^2 + 1$  dla $n=1, 2, \dots 15$
    \item pierwiastki kolejnych stopni od 2 do 100 z liczby 3
\end{enumerate}

\newcounter{whileloopx}\setcounter{whileloopx}{\arabic{zadanie}}
\textbf{\arabic{zadanie}.}\addtocounter{zadanie}{1} Dana jest lista: \texttt{l = [8, 14, 20, 7, 1, 8, 14, 12, 7, 14, 15, 14, 5, 11, 4]}. Napisz kod, który:
\begin{enumerate}[label=\arabic*.]
    \item wyszuka największą liczbę na tej liście
    \item wyszuka najmniejszą liczbę na tej liście
    \item obliczy sumę wszystkich liczb listy
    \item obliczy średnią tych liczb
    \item znajdzie drugą w kolejności największą liczbę na liście
    \item wyznaczy, ile razy \texttt{14} występuje na liście
    %~ \item z danej listy \texttt{l} tekstów wyszukuje najdłuższy. Możesz przyjąć, że taki element jest tylko jeden.
    %~ \item z danej listy \texttt{l} tekstów wyszukuje najkrótszy. Możesz przyjąć, że taki element jest tylko jeden.
\end{enumerate}
\textbf{Uwaga:} Nie wolno korzystać z odpowiednich funkcji Pythona.

\textbf{\arabic{zadanie}.}\addtocounter{zadanie}{1} Rozwiąż zadania \thewhileloop-\thewhileloopx\ za pomocą pętli \texttt{while}. Oczywiście, nie wolno korzystać z \verb|range()|.

KONIEC - while jest bez sensu do ostatniego zadania

\textbf{\arabic{zadanie}.}\addtocounter{zadanie}{1} Wykorzystaj  pętlę \verb|for| do wykonania następujących poleceń:
\begin{enumerate}[label=\arabic*.]
    \item sprawdź,
\end{enumerate}

\section{Pliki tekstowe}
W tym samym katalogu co program, znajduje się plik tekstowy o nazwie \texttt{dane.txt}.

\textbf{\arabic{zadanie}.}\addtocounter{zadanie}{1} Napisz kod, który wczytuje cały plik do pamięci i zapamiętuje go w zmiennej \texttt{dane}. Podaj kilka wariantów rozwiązania.

\textbf{\arabic{zadanie}.}\addtocounter{zadanie}{1} Napisz kod, który:
\begin{enumerate}[label=\arabic*.]
    \item wypisuje pierwszy wiersz tego pliku
    \item wypisuje pierwszy i drugi wiersz tego pliku
    \item poda liczbę wierszy pliku
    \item wypisze najdłuższy wiersz pliku (jeśli jest kilka o tej samej długości --- dowolny z nich)
    \item wypisze najkrótszy wiersz pliku --- jak wyżej
    \item wypisze ile wierszy pliku jest dłuższych niż 40 znaków
\end{enumerate}

\textbf{\arabic{zadanie}.}\addtocounter{zadanie}{1} Utwórz plik tekstowy o nazwie \texttt{witaj.txt} i zapisz do niego tekst: \texttt{Panie Janie, niech pan wstanie!}

\textbf{\arabic{zadanie}.}\addtocounter{zadanie}{1} Otwórz w trybie dopisywania plik utworzony w poprzednim zadaniu i dopisz na końcu tekst: \texttt{Panie Janie, rano wstań!}

\textbf{\arabic{zadanie}.}\addtocounter{zadanie}{1} Utwórz plik tekstowy o nazwie \texttt{iloczyny.txt} i w dziewięciu kolejnych wierszach zapisz kolejne iloczyny \texttt{1*1, 1*2, ..., 1*9}.

\textbf{\arabic{zadanie}.}\addtocounter{zadanie}{1} Utwórz plik tekstowy o nazwie \texttt{iloczyny.txt} i w osiemdziesięciu jeden kolejnych wierszach zapisz kolejne iloczyny \texttt{1*1, 1*2, ..., 1*9, 9*1, ..., 9*9}.

\textbf{\arabic{zadanie}.}\addtocounter{zadanie}{1} Utwórz plik tekstowy o nazwie \texttt{uczniowie.txt}. W pierwszym jego wierszu zapisz: \texttt{imię nazwisko klasa} i w trzech  kolejnych wierszach zapisz kolejno:
\begin{enumerate}[label=]
    \item \texttt{Anna Zielińska 3a}
    \item \texttt{Adam Pawłowski 3b}
    \item \texttt{Paulina Stefaniak 3d}
\end{enumerate}

\textbf{\arabic{zadanie}.}\addtocounter{zadanie}{1} Utwórz plik tekstowy i zapisz w nim kolejne sekwencje gwiazdek, jak niżej:
\begin{enumerate}[label=]
    \item \texttt{*}
    \item \texttt{**}
    \item \texttt{***}
    \item \texttt{\dots}
    \item $\underbrace{\texttt{**\dots *}}_{100}$
\end{enumerate}

\end{document}
